\begin{document}% document begins
  %
\subsection{Using Coordinates From Survey Plans}
  %
\subsubsection[How to use]{How to use Northing and Easting Coordinates Table}
  %
\paragraph[Use a Spreadsheet]{}\textbf{{\Large Using a spreadsheet to convert the dimensions}}
  %
\noindent To use Northing and Easting from survey plans: In a spreadsheet, adjust the data to be relative to the 1st point\\
  %
\noindent So if a survey gives you:\\

\begin{table}[htbp]
\centering
\resizebox{.5\linewidth}{!}{%
\begin{tabular}{|c|c|c|}
\hline
Pt&Northing&Easting\\ \hline
1&995.9952&9766.6\\
2&994.3049&9112\\
3&989.234&7150\\
4&1194.3099&9114\\
5&1193.266&8710.2059\\
6&1193.0954&8644.2016\\
...&...&...\\
32&1617.7856&8827.4296\\
\hline
\end{tabular}
}
\caption{Survey Plan Northing and Easting}
\end{table}

\clearpage

\noindent Calculate Relative North and Relative Easting of the points to Point 1 by subtracting the point 1 values from each of the other points.\\
%vspace{.1in}

\noindent Use formulas:

\begin{table}[htbp]
\centering
\resizebox{.75\linewidth}{!}{%
\begin{tabular}{|c|c|c|c|c|c|}
\hline

 &A&B&C&D&E\\\hline
1&Pt&Northing&Easting&Relative NS&Relative EW\\
2&1&995.9952&9766.6&0&0\\
3&2&994.3049&9112&=B3-B\textdollar2&=C3-C\textdollar2\\
4&3&989.234&7150&=B4-B\textdollar2&=C4-C\textdollar2\\
...&...&...&...&...&...\\
6&32&1617.7856&8827.4296&=B9-B\textdollar2&=C9-C\textdollar2\\
\hline
\end{tabular}
}
\caption{Survey Plan Northing and Easting}
\end{table}

Giving you:

\begin{table}[htbp]
\centering
\resizebox{.75\linewidth}{!}{%
\begin{tabular}{|c|c|c|c|c|c|}
\hline
 &A&B&C&D&E\\\hline
1&Pt&Northing&Easting&Relative NS&Relative EW\\
2&1&995.9952&9766.6&0&0\\
3&2&994.3049&9112&-1.6903&-654.6\\
4&3&989.234&7150&-6.7612&-2616.6\\
...&...&...&...&...&...\\
6&32&1617.7856&8827.4296&621.7904&-939.1704\\
\hline
\end{tabular}
}
\caption{Relative Northing and Easting}
\end{table}

So to place pt 32:\\

From pt 1:\\

Use distances 621.7904' N and 939.1704'W

\end{document}

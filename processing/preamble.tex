% Switches
%\renewcommand*{\bibname}{References} % renames the bibliography

%%%%%%%%%%%%%%%%%%%%%%%%%%%%%%%%%%%%%%%%%%%%%%%%%%%%%%%%%%%%%
% Tried and True Preamble Section
\usepackage{import} % Required for importing other .tex docs.  (import uses everything bw Begin and End Doc)
\usepackage{float} % Required for specifying the exact location of a figure or table
\usepackage{wrapfig} % Provides environment to wrap figures with text
\usepackage{graphicx} % Required for including images
\usepackage[pdftex,breaklinks,colorlinks=true,linkcolor=black,citecolor=blue,urlcolor=red,linktocpage=false,pagebackref=true,filecolor=magenta]{hyperref}% http://www.tug.org/applications/hyperref/manual.html#x1-100003.6
\usepackage{cite} % improves citation handling
\usepackage[toc,title,page]{appendix}
\usepackage{pdfpages} % enables loading a pdf into the doc
\usepackage{makeidx} % Enables index features
\usepackage{glossaries} % must be after hyperref
\usepackage{blindtext} % enables ipsum loerem "dummy" text blocks
\usepackage{enumitem} % enables control over enumerate, itemize, and description
\usepackage{xcolor}
\usepackage{marginnote}
%\usepackage{geometry}
\usepackage[top=1.5in, bottom=1.5in, outer=2.5in, inner=1.25in, heightrounded, marginparwidth=1.5in, marginparsep=.25in]{geometry} % Sets page geometry
\newcommand{\HRule}{\rule{\linewidth}{0.5mm}} % Command to make horizontal graphic lines
\newcommand{\marginText}{\fontfamily{cmr}\fontseries{sb}\color{red}\fontsize{9}{12}\selectfont}
\newcommand{\helv}{\fontfamily{phv}\fontseries{b}\fontsize{9}{11}\selectfont}
% font has encoding, family,  series{b}(bold), shape, size{(size)}{(baselineskip)}
%       % https://www.latex-project.org/help/documentation/fntguide.pdf
%       % https://www.overleaf.com/learn/latex/Font_typefaces
%
\graphicspath{{img/}{GIS_ChampionSection/img/}{awardsChapter/GIS_ChampionSection/img/}{brandPart/awardsChapter/GIS_ChampionSection/img/}{img/}{pairedProgSection/img/}{methodChapter/pairedProgSection/img/}{methodPart/methodChapter/pairedProgSection/img/}{documentationSection/img/}{methodChapter/documentationSection/img/}{methodPart/methodChapter/documentationSection/img/}{docStorageOrgSection/img/}{methodChapter/docStorageOrgSection/img/}{methodPart/methodChapter/docStorageOrgSection/img/}{QGisSection/img/}{toolsChapter/QGisSection/img/}{servicePart/toolsChapter/QGisSection/img/}{ESRISection/img/}{toolChapter/ESRISection/img/}{servicePart/toolChapter/ESRISection/img/}{../../../../source/}{../../source/}{servicePart/applicationsChapter/treasurerSection/img/}{servicePart/toolsChapter/gisAdminSection/img/}{gisAdminSection/img/}{servicePart/toolsChapter/bsaSupportSection/img/}{servicePart/toolsChapter/coreDataSection/img/}}
%
%%%%%%%%%%%%%%%%%%%%%%%%%%%%%%%%%%%%%%%%%%%%%%%%%%%%%%%%%%%%%
\usepackage{fancyhdr} % Adds control of headers and footers
%
%     Code Key: E : Even page, O : Odd page, L Left field,  C : Center field,
%               R : Right field, H : Header, F : Footer
%    \leftmark : Contains the Left argument of the last \markboth on the pages
%    \rightmark : Contains the Right Argument of the first \markboth of the page OR:
%                 the first /markright of the page
%
\renewcommand{\chaptermark}[1] % Redefinition of how chapter mark is displayed
{\markboth{\MakeUppercase{\thechapter.\ #1}}{}}
%
\renewcommand{\subsectionmark}[1] % Redefinition of how subsection mark is displayed
{\markright{\MakeUppercase{\thesubsection.\ #1}}}
%
\renewcommand{\headrulewidth}{0.5pt} % Line after header
\renewcommand{\footrulewidth}{0.5pt} % line before footer
%
\fancyhf{} % clear all header and footer fields
%
\fancyhead[LE,RO]{\helv \thepage} % Left Even and Right Odd headers
%                                 % \thepage macro displays the current page number
\fancyhead[LO]{\helv \leftmark} % Left Odd headers are set to /markboth
%                               % (chapters from \renewcommand{\chaptermark}[1]... )
\fancyhead[RE]{\helv \rightmark} % Right Even Headers
%                                % (subsection from \renewcommand{\subsectionmark}[1]...)
%
% Alternate fancy header options
%\renewcommand{\sectionmark}[1] % Redefinition of how section mark is displayed
%{\markright{\MakeUppercase{\thesection.\ #1}}}
%
%\fancyfoot[C]{\textbf{\thepage}} % Page number on bottom center
%\fancyfoot[EC,OC]{\helv \thesection} % Section number bottom center

%%%%%%%%%%%%%%%%%%%%%%%%%%%%%%%%%%%%%%%%%%%%%%%%%%%%%%%%%%
% Test area

%\setlength\parindent{0pt} % eliminates indents

%\tolerance=1500 % Number affects overfull and Underful box warnings
%\usepackage{caption}

%\setlist[description]{leftmargin=\parindent,labelindent=\parindent}
\newcommand*{\shortcenter}[1]{%
\sethangfrom{\noindent ##1}%
%\normalfont\boldmath\bfseries
\Largefont\boldmath\bfseries
\centering
\parbox{3in}{\centering #1}\par}

\setsubsubsecheadstyle{\shortcenter}

%\renewcommand{\setbeforesubsubsecskip}{-120pt plus -5pt minus -1pt}

\renewcommand{\midchapskip}{10pt} % defines vspace bw <Chapter#> and <Chapter Title> default is 20pt
\renewcommand{\afterchapskip}{20pt} % defines vspace bw <Chapter Title> and <Section Heading> defauklt uis 40pt

\documentclass[class=book , crop=false]{standalone}



%%%%%%%%%%%%%%%%%%%%%%%%%%%%%%%%%%%%%%%%%%%%%%%%%%%%%%%%%%%%%
% Tried and True Preamble Section
\usepackage{import} % Required for importing other .tex docs.  (import uses everything bw Begin and End Doc)
\usepackage{float} % Required for specifying the exact location of a figure or table
\usepackage{wrapfig} % Provides environment to wrap figures with text
\usepackage{graphicx} % Required for including images
\usepackage{cite} % improves citation handling
%\usepackage[toc,title,page]{appendix}
\usepackage{pdfpages} % enables loading a pdf into the doc
\usepackage{makeidx} % Enables index features
\usepackage{glossaries} % must be after hyperref
\usepackage{blindtext} % enables ipsum loerem "dummy" text blocks
\usepackage{enumitem} % enables control over enumerate, itemize, and description
\usepackage[dvipsnames]{xcolor}%black, blue, brown, cyan, darkgray, gray, green, lightgray, lime, magenta, olive, orange, pink, purple, red, teal, violet, white, yellow.
%
\definecolor{HeaderBlueA1}{RGB}{10, 91, 163}
\definecolor{HeaderBlueA2}{RGB}{101, 98, 174}
\definecolor{HeaderBlueA3}{RGB}{84, 125, 161}
\definecolor{HeaderBlueB}{RGB}{53, 98, 138}
\definecolor{HeaderBlueC}{RGB}{31,78,121}  % HeaderBlue1
\definecolor{HeaderBlueD}{RGB}{16, 60, 99}
\definecolor{HeaderBlueE}{RGB}{4, 40, 72 }

\definecolor{HeaderOrangeA}{RGB}{249, 161, 121}
\definecolor{HeaderOrangeB}{RGB}{214, 116, 72}
\definecolor{HeaderOrangeC}{RGB}{187, 84, 37}
\definecolor{HeaderOrangeD}{RGB}{152, 57, 14}
\definecolor{HeaderOrangeE}{RGB}{111, 35, 0}

\definecolor{HeaderGoldA}{RGB}{249, 218, 121}
\definecolor{HeaderGoldB}{RGB}{214, 180, 72}
\definecolor{HeaderGoldC}{RGB}{187, 151, 37}
\definecolor{HeaderGoldD}{RGB}{152, 119, 14}
\definecolor{HeaderGoldE}{RGB}{111, 84, 0}

% Color triad based on HeaderBlue1
% http://paletton.com/
% Blues:
% 84, 125, 161 lightest blue
% 53, 98, 138 light blue
% 31,78,121 HeaderBlue1
% 16, 60, 99 darker blue
% 4, 40, 72 darkest blue
% Oranges:
% 249, 161, 121 lightest orange
% 214, 116, 72 light orange
% 187, 84, 37 main orange
% 152, 57, 14 darker orange
% 111, 35, 0 dark orange
% Golds:
% 249, 218, 121 lightest goldenrod
% 214, 180, 72 light goldenrod
% 187, 151, 37 goldenrod
% 152, 119, 14 dark goldenrod
% 111, 84, 0 darkest goldenrod

\definecolor{HyperlinkBlue1}{RGB}{64,172,209}
\definecolor{graphicOrange}{RGB}{255,153,51}
% Header1 is Arial(body)(Caps) 14pt
% Header2 is Arial(body) 12pt
% Body is arial 9pt
\usepackage{marginnote}
%\usepackage{geometry}
\usepackage[top=1.5in, bottom=1.5in, outer=2.5in, inner=1.25in, heightrounded, marginparwidth=1.5in, marginparsep=.25in]{geometry} % Sets page geometry
\usepackage{fancyhdr} % Adds control of headers and footers
%%%%  Testing
\usepackage{pifont}
\usepackage{anyfontsize}
\usepackage{multicol}
%%%%%%%%%%%%%


%\newcommand{\localtextbulletone}{\textcolor{HeaderOrangeC}{\raisebox{.45ex}{\rule{.6ex}{.6ex}}}}

% ding characters specified in symbols-a4 pg12
\newcommand{\localtextbulletone}{\textcolor{HeaderOrangeC}{\ding{227}}}
\renewcommand{\labelitemi}{\localtextbulletone}

\newcommand{\HRule}{\rule{\linewidth}{0.5mm}} % Command to make horizontal graphic lines
\newcommand{\marginText}{\fontfamily{cmr}\fontseries{b}\color{red}\fontsize{9}{12}\selectfont}
\newcommand{\helv}{\fontfamily{phv}\fontseries{b}\fontsize{9}{11}\selectfont}
% font has encoding, family,  series{b}(bold), shape, size{(size)}{(baselineskip)}
%       % https://www.latex-project.org/help/documentation/fntguide.pdf
%       % https://www.overleaf.com/learn/latex/Font_typefaces
%
\graphicspath{{img/}{GIS_ChampionSection/img/}{awardsChapter/GIS_ChampionSection/img/}{brandPart/awardsChapter/GIS_ChampionSection/img/}{img/}{pairedProgSection/img/}{methodChapter/pairedProgSection/img/}{methodPart/methodChapter/pairedProgSection/img/}{documentationSection/img/}{methodChapter/documentationSection/img/}{methodPart/methodChapter/documentationSection/img/}{docStorageOrgSection/img/}{methodChapter/docStorageOrgSection/img/}{methodPart/methodChapter/docStorageOrgSection/img/}{QGisSection/img/}{toolsChapter/QGisSection/img/}{servicePart/toolsChapter/QGisSection/img/}{ESRISection/img/}{toolChapter/ESRISection/img/}{servicePart/toolChapter/ESRISection/img/}{../../../../source/}{../../source/}{servicePart/applicationsChapter/treasurerSection/img/}{servicePart/toolsChapter/gisAdminSection/img/}{gisAdminSection/img/}{servicePart/toolsChapter/bsaSupportSection/img/}{servicePart/toolsChapter/coreDataSection/img/}{../img}}
%
%%%%%%%%%%%%%%%%%%%%%%%%%%%%%%%%%%%%%%%%%%%%%%%%%%%%%%%%%%%%%
%\usepackage{fancyhdr} % Adds control of headers and footers
%
%     Code Key: E : Even page, O : Odd page, L Left field,  C : Center field,
%               R : Right field, H : Header, F : Footer
%    \leftmark : Contains the Left argument of the last \markboth on the pages
%    \rightmark : Contains the Right Argument of the first \markboth of the page OR:
%                 the first /markright of the page
%
\renewcommand{\chaptermark}[1] % Redefinition of how chapter mark is displayed
{\markboth{\MakeUppercase{\thechapter.\ #1}}{}}
%
\renewcommand{\subsectionmark}[1] % Redefinition of how subsection mark is displayed
{\markright{\MakeUppercase{\thesubsection.\ #1}}}
%
\renewcommand{\headrulewidth}{0.5pt} % Line after header
\renewcommand{\footrulewidth}{0.5pt} % line before footer
%
\fancyhf{} % clear all header and footer fields
%
\fancyhead[LE,RO]{\helv \thepage} % Left Even and Right Odd headers
%                                 % \thepage macro displays the current page number
\fancyhead[LO]{\helv \leftmark} % Left Odd headers are set to /markboth
%                               % (chapters from \renewcommand{\chaptermark}[1]... )
\fancyhead[RE]{\helv \rightmark} % Right Even Headers
%                                % (subsection from \renewcommand{\subsectionmark}[1]...)




%\fancyheadoffset[RE,LO,RO,LE]{.5\textwidth}
%

% Alternate fancy header options
%\renewcommand{\sectionmark}[1] % Redefinition of how section mark is displayed
%{\markright{\MakeUppercase{\thesection.\ #1}}}
%
% Alternate fancy footer options
%\fancyfoot[C]{\textbf{\thepage}} % Page number on bottom center
%\fancyfoot[EC,OC]{\helv \thesection} % Section number bottom center

%%%%%%%%%%%%%%%%%%%%%%%%%%%%%%%%%%%%%%%%%%%%%%%%%%%%%%%%%%
% Test area

%\setlength\parindent{0pt} % eliminates indents
%\tolerance=1500 % Number affects overfull and Underful box warnings
%\usepackage{caption}

%\setlist[description]{leftmargin=\parindent,labelindent=\parindent}
\newcommand{\titleGeometry}{\geometry{top=1in,bottom=1in,inner=1.5in,outer=1.5in}} % Sets geometry for first part}

\usepackage[pdftex,breaklinks,colorlinks=true,linkcolor=black,citecolor=blue,urlcolor=red,linktocpage=false,pagebackref=true,filecolor=magenta]{hyperref}% http://www.tug.org/applications/hyperref/manual.html#x1-100003.6
\hypersetup{pdfauthor={Allegan County GIS Services},% Page 248 in Latex Beinners Guide
pdftitle={What We Do},
pdfsubject={Documentation of everything},
pdfkeywords={documentation,gis}}


%\graphicspath{{img/}{QGisSection/img/}{toolsChapter/QGisSection/img/}{servicePart/toolsChapter/QGisSection/img/}}

\title{ % create title page
\HRule % Horizontal Line added
\\[.4cm] % space
\begin{figure}[H] % included image
\begin{center}	% centered horizontally
	\includegraphics[scale=.45]{GIS_Logo_better.jpg}
	\end{center}
	\end{figure}
	\Huge \bfseries Parcel Editing with COGO Tools in QGIS % Title text
\HRule \\[.4cm] % Horizontal Line added
}  % closing brace for title

\author{\Large Allegan County GIS \\\Large www.allegancounty.org/gis} % defines author

%-------------------------------------------------------------------------
%		Front Section											         %
%-------------------------------------------------------------------------
\begin{document}% Document Begins

%\maketitle % creates title page here when this page is compiled alone

\subsection{Using COGO Tools in QGIS}
	\medskip 
	\subsubsection{\Large Set up the Azimuth and Distance Plugin \\\small(Azd Plugin).}
	%\subparagraph{After starting QGIS in the basemap}
	\medskip
		\large In the Plugins drop down(1), \large under the topography group\\\Large select the \textbf {Azd Plugin(2)}(see fig.).
		\begin{figure}[H] % included image
		\begin{center}
		\includegraphics[scale=.30]{1.png}
		\end{center}
		\caption{launch plugin}
		\end{figure}
\clearpage
		
		\large Note here which layer is active (see fig.).
		\begin{figure}[H] % included image
		\begin{center}
		\includegraphics[scale=.26]{2.png}
		\end{center}
		\caption{check active layer}
		\end{figure}

%\clearpage

		If necessary, left click the layer \textbf {\emph{traverse 1}} in Layer Panel to activate it(see fig.).
			\begin{figure}[H] % Example of including images
			\begin{center}
			\includegraphics[scale=.2]{3.png}
			\end{center}
			\caption{activate layer}
			\end{figure}

\clearpage

		\paragraph{Configure Options}
			\large On Options Tab: Select Boundary, Bearing, Feet, and Degree radio buttons.
			\begin{figure}[H]
			\begin{center}
			\includegraphics[scale=.25]{4.png}
			\end{center}
			\caption{Plugin Options}
			\end{figure}
\clearpage

		\paragraph{Using the tool}
			\large Boundary descriptions are entered into the Drawing Tab. Azimuth (bearing) and Distance are the important boxes (Set Offset = 0 and Zenith = 90 and ignore)(see below).
			\begin{figure}[H]
			\begin{center}
			\includegraphics[scale=.25]{5.png}
			\end{center}
			\caption{Entering Bounds}
			\end{figure}

\clearpage

	\subsubsection{Configure editing environment}
	\Large Use Settings Dropdown and Snapping Options to enable snapping to Sections, Quarter Sections, and or Parcels if desired (see fig.).

		\begin{figure}[H]
		\begin{center}
		\includegraphics[scale=.20]{7.png}
		\end{center}
		\caption{Configure editing environment}
		\end{figure}

\clearpage

	\subsubsection{\Large Locate Point of Commencement}
		To get to the Point of Commencement,
		\medskip

		Use \textbf{any combination} of the following methods:
		\begin{itemize}
		\item{Using Reference Layer}
		\item{Using Measuring Tool}
		\item{Search by Parcel Number \small(Search Layers Plugin)}
		\item{Draw COGO lines \small(Azd Plugin)}\small(as described earlier)
		\end{itemize}

		\paragraph{Using Reference Layer}
			\large Use reference layers; Units, AC\_SectionsQu, Sections, and Parcels.  Toggle layers on and off in Layers Panel and zoom in and out with mouse wheel.
			
\clearpage
		\paragraph{Using Measuring Tool}
			\large Use the measuring tool, make sure to set units to feet.  To exit current measurement right click (see fig.).
				\begin{figure}[H]
				\begin{center}
				\includegraphics[scale=.20]{6.png}
				\end{center}
				\caption{Measuring Tool}
				\end{figure}	
						
\clearpage			
		\paragraph{Search by Parcel Number}
			\small (Search Layers Plugin.)
			\subparagraph{}
				To Launch Search Layers Plugin:\\In Plugins dropdown:\\Enable the \textbf{Search Layers} Plugin. (see fig.)		
				\begin{figure}[H]
				\begin{center}
				\includegraphics[scale=.20]{SearchLayers1b.PNG}
				\end{center}
				\caption{Search Layers Plugin}
				\end{figure}
			\bigskip
				Enter parcel number {\tiny (with dashes)}, Set layers, and set search field.(see fig.) 
				\begin{figure}[H]						
				\begin{center}
				\includegraphics[scale=.20]{SearchLayers.png}
				\end{center}
				\caption{Search Layers Setup}
				\end{figure}
				
							
\clearpage			
			

\end{document}

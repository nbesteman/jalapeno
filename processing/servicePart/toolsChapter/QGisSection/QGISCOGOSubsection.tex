  %
  %
  %
%+++++++++++++++++++++++++++++++++++++++++++++++++++++++++++++++++++
%    To Do:
%
%
%
%-------------------------------------------------------------------
  %
  % OPTIONAL PREAMBLE FOR LOCAL COMPILE  %
    %
%   \def\titlename{Parcel Editing with COGO Tools in QGIS}
%   \def\authorName{Allegan County GIS Services}
%   \def\pdfTitle{Parcel Editing with COGO Tools in QGIS}
%   \def\pdfSubject{GIS Tools} %
%   \def\pdfKeywords{mobile,gis}
%    %
%   %+++++++++++++++++++++++++++++++++++++++++++++++++++++++++++++++++++
%  Used in Subsections that are capable of being compiled alone
%
%
%-------------------------------------------------------------------
\documentclass[letterpaper, 12pt]{memoir}
 %


%%%%%%%%%%%%%%%%%%%%%%%%%%%%%%%%%%%%%%%%%%%%%%%%%%%%%%%%%%%%%
% Tried and True Preamble Section
\usepackage{import} % Required for importing other .tex docs.  (import uses everything bw Begin and End Doc)
\usepackage{float} % Required for specifying the exact location of a figure or table
\usepackage{wrapfig} % Provides environment to wrap figures with text
\usepackage{graphicx} % Required for including images
\usepackage{cite} % improves citation handling
%\usepackage[toc,title,page]{appendix}
\usepackage{pdfpages} % enables loading a pdf into the doc
\usepackage{makeidx} % Enables index features
\usepackage{glossaries} % must be after hyperref
\usepackage{blindtext} % enables ipsum loerem "dummy" text blocks
\usepackage{enumitem} % enables control over enumerate, itemize, and description
\usepackage[dvipsnames]{xcolor}%black, blue, brown, cyan, darkgray, gray, green, lightgray, lime, magenta, olive, orange, pink, purple, red, teal, violet, white, yellow.
%
\definecolor{HeaderBlueA1}{RGB}{10, 91, 163}
\definecolor{HeaderBlueA2}{RGB}{101, 98, 174}
\definecolor{HeaderBlueA3}{RGB}{84, 125, 161}
\definecolor{HeaderBlueB}{RGB}{53, 98, 138}
\definecolor{HeaderBlueC}{RGB}{31,78,121}  % HeaderBlue1
\definecolor{HeaderBlueD}{RGB}{16, 60, 99}
\definecolor{HeaderBlueE}{RGB}{4, 40, 72 }

\definecolor{HeaderOrangeA}{RGB}{249, 161, 121}
\definecolor{HeaderOrangeB}{RGB}{214, 116, 72}
\definecolor{HeaderOrangeC}{RGB}{187, 84, 37}
\definecolor{HeaderOrangeD}{RGB}{152, 57, 14}
\definecolor{HeaderOrangeE}{RGB}{111, 35, 0}

\definecolor{HeaderGoldA}{RGB}{249, 218, 121}
\definecolor{HeaderGoldB}{RGB}{214, 180, 72}
\definecolor{HeaderGoldC}{RGB}{187, 151, 37}
\definecolor{HeaderGoldD}{RGB}{152, 119, 14}
\definecolor{HeaderGoldE}{RGB}{111, 84, 0}

% Color triad based on HeaderBlue1
% http://paletton.com/
% Blues:
% 84, 125, 161 lightest blue
% 53, 98, 138 light blue
% 31,78,121 HeaderBlue1
% 16, 60, 99 darker blue
% 4, 40, 72 darkest blue
% Oranges:
% 249, 161, 121 lightest orange
% 214, 116, 72 light orange
% 187, 84, 37 main orange
% 152, 57, 14 darker orange
% 111, 35, 0 dark orange
% Golds:
% 249, 218, 121 lightest goldenrod
% 214, 180, 72 light goldenrod
% 187, 151, 37 goldenrod
% 152, 119, 14 dark goldenrod
% 111, 84, 0 darkest goldenrod

\definecolor{HyperlinkBlue1}{RGB}{64,172,209}
\definecolor{graphicOrange}{RGB}{255,153,51}
% Header1 is Arial(body)(Caps) 14pt
% Header2 is Arial(body) 12pt
% Body is arial 9pt
\usepackage{marginnote}
%\usepackage{geometry}
\usepackage[top=1.5in, bottom=1.5in, outer=2.5in, inner=1.25in, heightrounded, marginparwidth=1.5in, marginparsep=.25in]{geometry} % Sets page geometry
\usepackage{fancyhdr} % Adds control of headers and footers
%%%%  Testing
\usepackage{pifont}
\usepackage{anyfontsize}
\usepackage{multicol}
%%%%%%%%%%%%%


%\newcommand{\localtextbulletone}{\textcolor{HeaderOrangeC}{\raisebox{.45ex}{\rule{.6ex}{.6ex}}}}

% ding characters specified in symbols-a4 pg12
\newcommand{\localtextbulletone}{\textcolor{HeaderOrangeC}{\ding{227}}}
\renewcommand{\labelitemi}{\localtextbulletone}

\newcommand{\HRule}{\rule{\linewidth}{0.5mm}} % Command to make horizontal graphic lines
\newcommand{\marginText}{\fontfamily{cmr}\fontseries{b}\color{red}\fontsize{9}{12}\selectfont}
\newcommand{\helv}{\fontfamily{phv}\fontseries{b}\fontsize{9}{11}\selectfont}
% font has encoding, family,  series{b}(bold), shape, size{(size)}{(baselineskip)}
%       % https://www.latex-project.org/help/documentation/fntguide.pdf
%       % https://www.overleaf.com/learn/latex/Font_typefaces
%
\graphicspath{{img/}{GIS_ChampionSection/img/}{awardsChapter/GIS_ChampionSection/img/}{brandPart/awardsChapter/GIS_ChampionSection/img/}{img/}{pairedProgSection/img/}{methodChapter/pairedProgSection/img/}{methodPart/methodChapter/pairedProgSection/img/}{documentationSection/img/}{methodChapter/documentationSection/img/}{methodPart/methodChapter/documentationSection/img/}{docStorageOrgSection/img/}{methodChapter/docStorageOrgSection/img/}{methodPart/methodChapter/docStorageOrgSection/img/}{QGisSection/img/}{toolsChapter/QGisSection/img/}{servicePart/toolsChapter/QGisSection/img/}{ESRISection/img/}{toolChapter/ESRISection/img/}{servicePart/toolChapter/ESRISection/img/}{../../../../source/}{../../source/}{servicePart/applicationsChapter/treasurerSection/img/}{servicePart/toolsChapter/gisAdminSection/img/}{gisAdminSection/img/}{servicePart/toolsChapter/bsaSupportSection/img/}{servicePart/toolsChapter/coreDataSection/img/}{../img}}
%
%%%%%%%%%%%%%%%%%%%%%%%%%%%%%%%%%%%%%%%%%%%%%%%%%%%%%%%%%%%%%
%\usepackage{fancyhdr} % Adds control of headers and footers
%
%     Code Key: E : Even page, O : Odd page, L Left field,  C : Center field,
%               R : Right field, H : Header, F : Footer
%    \leftmark : Contains the Left argument of the last \markboth on the pages
%    \rightmark : Contains the Right Argument of the first \markboth of the page OR:
%                 the first /markright of the page
%
\renewcommand{\chaptermark}[1] % Redefinition of how chapter mark is displayed
{\markboth{\MakeUppercase{\thechapter.\ #1}}{}}
%
\renewcommand{\subsectionmark}[1] % Redefinition of how subsection mark is displayed
{\markright{\MakeUppercase{\thesubsection.\ #1}}}
%
\renewcommand{\headrulewidth}{0.5pt} % Line after header
\renewcommand{\footrulewidth}{0.5pt} % line before footer
%
\fancyhf{} % clear all header and footer fields
%
\fancyhead[LE,RO]{\helv \thepage} % Left Even and Right Odd headers
%                                 % \thepage macro displays the current page number
\fancyhead[LO]{\helv \leftmark} % Left Odd headers are set to /markboth
%                               % (chapters from \renewcommand{\chaptermark}[1]... )
\fancyhead[RE]{\helv \rightmark} % Right Even Headers
%                                % (subsection from \renewcommand{\subsectionmark}[1]...)




%\fancyheadoffset[RE,LO,RO,LE]{.5\textwidth}
%

% Alternate fancy header options
%\renewcommand{\sectionmark}[1] % Redefinition of how section mark is displayed
%{\markright{\MakeUppercase{\thesection.\ #1}}}
%
% Alternate fancy footer options
%\fancyfoot[C]{\textbf{\thepage}} % Page number on bottom center
%\fancyfoot[EC,OC]{\helv \thesection} % Section number bottom center

%%%%%%%%%%%%%%%%%%%%%%%%%%%%%%%%%%%%%%%%%%%%%%%%%%%%%%%%%%
% Test area

%\setlength\parindent{0pt} % eliminates indents
%\tolerance=1500 % Number affects overfull and Underful box warnings
%\usepackage{caption}

%\setlist[description]{leftmargin=\parindent,labelindent=\parindent}
\newcommand{\titleGeometry}{\geometry{top=1in,bottom=1in,inner=1.5in,outer=1.5in}} % Sets geometry for first part}

\usepackage[pdftex,breaklinks,colorlinks=true,linkcolor=black,citecolor=blue,urlcolor=red,linktocpage=false,pagebackref=true,filecolor=magenta]{hyperref}% http://www.tug.org/applications/hyperref/manual.html#x1-100003.6
\hypersetup{pdfauthor={Allegan County GIS Services},% Page 248 in Latex Beinners Guide
pdftitle={What We Do},
pdfsubject={Documentation of everything},
pdfkeywords={documentation,gis}}

 %
\graphicspath{{img/}{GIS_ChampionSection/img/}{awardsChapter/GIS_ChampionSection/img/}{brandPart/awardsChapter/GIS_ChampionSection/img/}{img/}{pairedProgSection/img/}{methodChapter/pairedProgSection/img/}{methodPart/methodChapter/pairedProgSection/img/}{documentationSection/img/}{methodChapter/documentationSection/img/}{methodPart/methodChapter/documentationSection/img/}{docStorageOrgSection/img/}{methodChapter/docStorageOrgSection/img/}{methodPart/methodChapter/docStorageOrgSection/img/}{QGisSection/img/}{toolsChapter/QGisSection/img/}{servicePart/toolsChapter/QGisSection/img/}{ESRISection/img/}{toolChapter/ESRISection/img/}{servicePart/toolChapter/ESRISection/img/}{../../../../source/}{../../source/}{servicePart/applicationsChapter/treasurerSection/img/}{servicePart/toolsChapter/gisAdminSection/img/}{gisAdminSection/img/}{servicePart/toolsChapter/bsaSupportSection/img/}{servicePart/toolsChapter/coreDataSection/img/}{../img}}

 %
%   This script does not require Memoir Class
%%%%%%%%%%%%%%%%%%%%%%%%%%%%%%%%%%%%%%%%%%%%%%%%%%%%%%%%%%%%
%+++++++++++++++++++++++++++++++++++++++++++++++++++++++++++++++++++
% Custom Color pallette
    % Blues
\definecolor{HeaderBlueA}{RGB}{85,125,161}
\definecolor{HeaderBlueB}{RGB}{53, 98, 138}
\definecolor{HeaderBlueC}{RGB}{31,78,121}  % HeaderBlue1
\definecolor{HeaderBlueD}{RGB}{16, 60, 99}
\definecolor{HeaderBlueE}{RGB}{4, 40, 72 }
    % Oranges
\definecolor{HeaderOrangeA}{RGB}{249, 161, 121}
\definecolor{HeaderOrangeB}{RGB}{214, 116, 72}
\definecolor{HeaderOrangeC}{RGB}{187, 84, 37}
\definecolor{HeaderOrangeD}{RGB}{152, 57, 14}
\definecolor{HeaderOrangeE}{RGB}{111, 35, 0}
    % Golds
\definecolor{HeaderGoldA}{RGB}{249, 218, 121}
\definecolor{HeaderGoldB}{RGB}{214, 180, 72}
\definecolor{HeaderGoldC}{RGB}{187, 151, 37}
\definecolor{HeaderGoldD}{RGB}{152, 119, 14}
\definecolor{HeaderGoldE}{RGB}{111, 84, 0}
    % Greens
    % option 1
% \definecolor{HeaderGreenA}{RGB}{14, 219, 76}
% \definecolor{HeaderGreenB}{RGB}{23, 179, 70}
% \definecolor{HeaderGreenC}{RGB}{31, 131, 61}
% \definecolor{HeaderGreenD}{RGB}{30, 92, 49}
% \definecolor{HeaderGreenE}{RGB}{21, 50, 30}
    % option 2 (triad of HeaderBlueC)
\definecolor{HeaderGreenA}{RGB}{82, 169, 136}
\definecolor{HeaderGreenB}{RGB}{49, 145, 109}
\definecolor{HeaderGreenC}{RGB}{25, 126, 88}
\definecolor{HeaderGreenD}{RGB}{10, 103, 68}
\definecolor{HeaderGreenE}{RGB}{0, 75, 47}
% Color triad based on HeaderBlue1
% http://paletton.com/

\definecolor{HyperlinkBlue1}{RGB}{64,172,209}
\definecolor{graphicOrange}{RGB}{255,153,51}
%-------------------------------------------------------------------

 %
%  Requires Memoir Class
%%%%%%%%%%%%%%%%%%%%%%%%%%%%%%%%%%%%%%%%%%%%%%%%%%%%%%%%%%%%%%%%%%%
    % Sets sectioning Layout Properties (SPACING AND TYPEFACE PROPERTIES)
      % FOR HEADINGS OF SECTIONS, SUBSECTIONS, SUBSUBSECTIONS, PARAGRAPH AND SUBPARAGRAPH
      %
\setsecnumdepth{subsection} % turns on subsec numbering
      %
%+++++++++++++++++++++++++++++++++++++++++++++++++++++++++++++++++++
%  SET HEAD STYLES
\setsecheadstyle{\scshape\color{HeaderOrangeA}} % default is \Large\bfseries
    %\setsecheadstyle{\scshape\color{HeaderOrangeB}}
    %
    % Subsection
\setsubsecheadstyle{\large\scshape\color{HeaderOrangeB}} % default is \large\bfseries
    %\setsubsecheadstyle{\huge\centering\color{HeaderOrangeE}}%
    % Subsubsection
\setsubsubsecheadstyle{\Huge\scshape\centering\color{HeaderBlueE}} % default is \bfseries
    %\setsubsubsecheadstyle{\LARGE\scshape\centering\color{HeaderBlueC}}%   1
    % Paragraph
\setparaheadstyle{\bfseries\Large\color{HeaderBlueC}}%
     %\setparaheadstyle{\Large\color{HeaderBlueA}}%2
     % Subparagraph
     %\setsubparaheadstyle{\large\color{HeaderBlueB}}%
\setsubparaheadstyle{\bfseries\large\color{HeaderOrangeB}}%
%-------------------------------------------------------------------
  %
%+++++++++++++++++++++++++++++++++++++++++++++++++++++++++++++++++++
  %  SET BEFORE SKIPS (BW the title and the text)
     % \setbeforeSskip{<skip>} pg .95
     % The absolute value of the <skip> length argument is the space to leave above the heading.
     % If the actual value is negative then the first line after the heading will not be indented.
\setbeforesecskip{-35pt plus -10pt minus -2pt}
\setbeforesubsecskip{-32.5pt plus -10pt minus -2pt}
\setbeforesubsubsecskip{-32.5pt plus -10pt minus -2pt}
\setbeforeparaskip{-32.5pt plus -10pt minus -2pt}
\setbeforesubparaskip{-5pt plus -3pt minus -1pt}
%-------------------------------------------------------------------
  %
%+++++++++++++++++++++++++++++++++++++++++++++++++++++++++++++++++++
  %  SET BEFORE INDENTS
    % \setSindent{<length>}
    % The value of the length argument is the indentation of the heading (number and
    % title) from the lefthand margin. This is normally 0pt.
\setsubparaindent{0pt}
%-------------------------------------------------------------------
  %
%+++++++++++++++++++++++++++++++++++++++++++++++++++++++++++++++++++
  %  SET AFTER SKIPS (After the text)
    % \setafterSskip{<skip>} pg.96
    % If the value of the <skip> length argument is positive it is the space to leave between the
    % display heading and the following text. If it is negative, then the heading will be runin
    % and the value is the horizontal space between the end of the heading and the following text.
\setaftersecskip{.15in}
\setaftersubsecskip{.15in}
\setaftersubsubsecskip{.1in}
\setafterparaskip{.05in}
\setaftersubparaskip{.05in}
%-------------------------------------------------------------------
  %
%+++++++++++++++++++++++++++++++++++++++++++++++++++++++++++++++++++
  %  REDEFINE PART NAME AND NUMBER FONT
\renewcommand{\partnamefont}{\color{HeaderOrangeE}\Huge\normalfont}
\renewcommand{\partnumfont}{\color{HeaderOrangeE}\Huge\normalfont}
%-------------------------------------------------------------------
%
\setlength\columnsep{.3in} % Space BW columns in 2 col mode

 %
%   This script requires Memoir Class
%%%%%%%%%%%%%%%%%%%%%%%%%%%%%%%%%%%%%%%%%%%%%%%%%%%%%%%%%%%%
    %  Page Styles
    %  In LATEX the page style refers to the part of the page building mechanism that
    %   attaches the headers and footers to the actual page.
%%%%%%%%%%%%%%%%%%%%%%%%%%%%%%%%%%%%%%%%%%%%%%%%%%%%%%%%%%%%
%  This block defines jalapenoPageStyleA page style
\makepagestyle{jalapenoPageStyleA}
    % headers
\makeheadrule {jalapenoPageStyleA}{\textwidth}{\normalrulethickness}
\makeheadfootruleprefix{jalapenoPageStyleA}{\color{HeaderOrangeE}}{\color{HeaderOrangeE}}
\makeevenhead {jalapenoPageStyleA}{\helv \thepage}{} {\helv \rightmark}
\makeoddhead {jalapenoPageStyleA}{\helv \leftmark}{}{\helv \thepage}
    % footers
\makefootrule {jalapenoPageStyleA}{\textwidth}{\normalrulethickness}{\footruleskip}
\makeevenfoot {jalapenoPageStyleA}{}{} {}
\makeoddfoot {jalapenoPageStyleA}{}{} {}
    %
%%%%%%%%%%
    %
%  This block defines marks for jalapenoPageStyleA page style
    %  See page 11 of Page Styles in Memoir
    %
\makeatletter % because of \@chapapp
%
\makepsmarks {jalapenoPageStyleA}{
%\nouppercaseheads
\createmark {chapter} {both} {shownumber}{\@chapapp\ }{. \ } % set leftmark to chapter
%\createmark {section} {right}{shownumber}{} {. \ }
\createmark {subsection} {right}{shownumber}{} {. \ } % set rightmark to subsection
%\createmark {subsubsection}{right}{shownumber}{} {. \ } % set rightmark to subsubsection
\createplainmark {toc} {both} {\contentsname}
\createplainmark {lof} {both} {\listfigurename}
\createplainmark {lot} {both} {\listtablename}
\createplainmark {bib} {both} {\bibname}
\createplainmark {index} {both} {\indexname}
\createplainmark {glossary} {both} {\glossaryname}

}
\makeatother
%  End of jalapenoPageStyleA pagestyle
%%%%%%%%%%%%%%%%%%%%%%%%%%%%%%%%%%%%%%%%%%%%%%%%%%%%%%%%%%%%

%%%%%%%%%%%%%%%%%%%%%%%%%%%%%%%%%%%%%%%%%%%%%%%%%%%%%%%%%%%%
%  This block defines emptyPageStyle page style
\makepagestyle{emptyPageStyle}
    % headers
\makeevenhead {emptyPageStyle}{}{}{}
\makeoddhead {emptyPageStyle}{}{}{}
    % footers
\makefootrule {emptyPageStyle}{}{}{}
\makeevenfoot {emptyPageStyle}{}{}{}
\makeoddfoot {emptyPageStyle}{}{}{}
    %
%%%%%%%%%%

 %
%   This script requires Memoir Class
%%%%%%%%%%%%%%%%%%%%%%%%%%%%%%%%%%%%%%%%%%%%%%%%%%%%%%%%%%%%%%%%%%%%%%%%%%%%
  % beginning od customchpstyle
    %
\makeatletter % handler for @
    %
\makechapterstyle{jalapenoChapterStyle}{%
    %
    \newlength{\chapColorbarheight}
    \setlength{\chapColorbarheight}{.06in}		% Setting the height of the bar
    %
    \newlength{\chapnumboxlength}
    \setlength{\chapnumboxlength}{.75in}	% Setting the length of the box containing chapter number
    %
    \newlength{\leftbarlength}
    \setlength{\leftbarlength}{.5in}  		% Setting length of the bar left to the chapter number
    \setlength{\afterchapskip}{.5in}
    \renewcommand*{\chapterheadstart}{\vspace*{.25in}} % vspace from top of page
    \renewcommand*{\afterchapternum}{\vspace*{-.9in}} % neg moves chapter title above chapter number
    \renewcommand*{\chapnumfont}{\slshape} % Number Font
    \renewcommand*{\chaptitlefont}{\flushright\color{HeaderOrangeE}\slshape\huge}
    \renewcommand*{\printchaptername}{} % sets the text in brackets before the number
    %\makeatletter
    \newcommand*{\thickrulefill}{\leavevmode \leaders \hrule height \chapColorbarheight \hfill \kern \z@} % rightbar length setter
    %\makeatother
    \renewcommand*{\printchapternum}{%
        %\makebox[0pt][l]{%
        \color{HeaderOrangeE}
        \rule{\leftbarlength}{\chapColorbarheight} % Left bar
        \quad % hspace
        \resizebox{!}{\chapnumboxlength}
        {\chapnumfont \thechapter} % Chapter name and number
        \quad % hspace
        \color{HeaderOrangeE}\thickrulefill % implements thickrulefill
     } %
     \makeatother % closes handler for @
} % End of customchpstyle
 %\makeatother % closes handler for @
%%%%%%%%%%%%%%%%%%%%%%%%%%%%%%%%%%%%%%%%%%%%%%%%%%%%%%%%%%%%%%%%%%%%%%%%%%%%

 %
%\newglossaryentry{ex}{name={sample},description={an example}}


\newglossaryentry{projection}{name={map projection},description={Representing a sphere on a flat surface}}
 %
\newlength\drop
\makeatletter
\newcommand*\titleM{\begingroup% Misericords, T&H p 153
\setlength\drop{0.08\textheight}
\centering
\vspace*{\drop}
{\protect\HRule}
\begin{figure}[H] % included image
\begin{center}	% centered horizontally
\includegraphics[scale=.6]{GIS_Logo_better.jpg}
\end{center}
\end{figure}
\vspace{-.1in}
{\Huge\bfseries\titlename}\\[2\baselineskip]
{\protect\HRule}\\[\baselineskip]
{\small\scshape www.allegancounty.org/gis}\\[2\baselineskip]
{\small\scshape \@date}\par
\endgroup}
\makeatother
  % inputs common title
  % SET PDF METADATA  %
\hypersetup{pdfauthor={\authorName},
pdftitle={\pdfTitle}, %  Sets PDF properties
pdfsubject={\pdfSubject},
pdfkeywords={\pdfKeywords}}
%-------------------------------------------------------------------
  % TOC DEPTH  %
\setcounter{tocdepth}{4}  % Sets Table of Contents level to show subsections, sections, chapters, and parts(DEFAULT)
 %
%+++++++++++++++++++++++++++++++++++++++++++++++++++++++++++++++++++
%    SET TEXT BLOCK AND MARGINS FOR COVER PAGE  %
\setmarginnotes{.1in}{.4in}{.1in}
\setlrmarginsandblock{*}{0.18\paperwidth}{1} % Left right ratio
\setulmarginsandblock{1in}{1in}{*}
\checkandfixthelayout
\makeatletter
\ch@ngetext
\makeatother
  %+++++++++++++++++++++++++++++++++++++++++++++++++++++++++++++++++++
	%  Front Section
  %-------------------------------------------------------------------
\begin{document}% document begins
   %
\frontmatter % turns off chapter numbering and uses roman numerals for page numbers
   %
\pagestyle{empty} % Clear headers and footers for TOC
   %
\begin{titlingpage}
   %
\titleM  % Inputs titleM
   %
\end{titlingpage}
   %
%+++++++++++++++++++++++++++++++++++++++++++++++++++++++++++++++++++
%    SET TEXT BLOCK AND MARGINS FOR MAIN DOCUMENT PAGES   %
\setmarginnotes{.1in}{.4in}{.1in}
\setlrmarginsandblock{*}{0.18\paperwidth}{.75} % Left right ratio
\setulmarginsandblock{1in}{1in}{*}
\checkandfixthelayout
\makeatletter
\ch@ngetext
\makeatother
%-------------------------------------------------------------------
\tableofcontents % creates TOC
  %
%+++++++++++++++++++++++++++++++++++++++++++++++++++++++++++++++++++
%		Main Section
%-------------------------------------------------------------------
\mainmatter % turns on chapter numbering, resets page numbering and uses arabic numerals for page numbers
  %
\chapterstyle{jalapenoChapterStyle} % custom from chapterStyles.tex
  %
\pagestyle{jalapenoPageStyleA} % custom from pageStyles.tex

%\setlength{\parindent}{24pt}
  %-------------------------------------------------------------------
      %
\begin{document}% document begins
      %
      %
  %-------------------------------------------------------------------
      %
\subsection{COGO Tools in QGIS}
  %

\subsubsection{Tool Summary}
  %
Transfers of real property typically involve a Metes and Bounds description:
  %
\begin{figure}[h!]
\centering
    \includegraphics[width=1\textwidth]{DescriptionFromDeed.PNG}
\vspace{-.2in}

\caption{Description From Deed}
\end{figure}
\begin{adjmulticols}{2}{\innerMar}{\outerMar}
  %
\paragraph{Background}
  %

In GIS, \textit{Coordinate Geometry} or \textbf{COGO} tools convert written descriptions of real property into digital map features.\\

\noindent Users in several county departments use COGO tools in their regular workflow.\\

\paragraph{Why the Tool is Needed}
  %
A tool is needed to convert between written descriptions of real property and digital map data.\\

\noindent The COGO tools in ArcGIS require an advanced license.
  %
\paragraph{Who the Tool is For}
  %
A user with QGIS installed locally and the ability to make a basic map.
  %
\paragraph{Takeaways}
  %
QGIS is an open source GIS without a built in COGO toolset.\\

\noindent The Azimuth and Distance Plugin provides the COGO functionality in QGIS.
  %
\end{adjmulticols}
  %
\vspace{1in}

\subparagraph{Following are instructions for using QGIS for COGO}

\clearpage
  %
  %
\paragraph{To use COGO tools in QGIS, follow these steps}
  %
\vspace{.2in}

\stepcounter{stepCount}

\subparagraph*{{\LARGE Step \thestepCount:}\\Launch and Configure the Azimuth and Distance Plugin}

\textasteriskcentered{\scriptsize Plugin installation is covered in a separate document.}

\begin{figure}[H]
\centering
     \includegraphics[width=1\textwidth]{cogoIcon.PNG}
\vspace{-.2in}

\caption{COGO Icon}
\end{figure}
  %
\vspace{-.2in}

\textasteriskcentered{\scriptsize This tool draws in a temporary layer or in an active map layer.}
  %
\vspace{.3in}

\noindent {\large Select \textbf{\fbox{traverse1}} as active layer in the tool.}

\begin{figure}[H] % included image
\centering
    \includegraphics[width=1\textwidth]{checkActiveLayer.png}
\vspace{-.3in}

\caption{Check Active Layer}
\end{figure}
  %
\clearpage
  %
\noindent\textbf{Configure Options in Plugin}
\vspace{.2in}

\large On the \textbf{\fbox{Options} Tab:} Select these radio buttons;
  %
\begin{itemize}
  %
\item \fbox{Boundary}
  %
\item \fbox{Bearing}
  %
\item \fbox{Feet}
  %
\item \fbox{Degree}
  %
\end{itemize}
  %
\begin{figure}[H]
\centering
    \includegraphics[width=1\textwidth]{pluginOptions.png}
\vspace{-.2in}

\caption{Plugin Options}
\end{figure}
  %
\clearpage

\stepcounter{stepCount}

\subparagraph*{{\LARGE Step \thestepCount:} Activate traverse layer in map}
  %
\vspace{.2in}

\textasteriskcentered{\scriptsize For a map layer to be editable, it must be activated in the Layers Panel.}
  %
\vspace{.2in}

\noindent{\scriptsize (If necessary)} left click the layer \textbf {\fbox{traverse1}} in Layer Panel to activate it.
  %
\vspace{.2in}

\begin{figure}[H] % Example of including images
\centering
    \includegraphics[width=1\textwidth]{activateLayer.png}
\vspace{-.1in}

\caption{activate layer}
\end{figure}

\clearpage

\stepcounter{stepCount}

\subparagraph*{{\LARGE Step \thestepCount:} Locate the Point of Commencement}
  %
\vspace{.2in}

\noindent To get to the Point of Commencement,
\vspace{.2in}

\noindent Use \textbf{any combination} of the following methods:
  %
\vspace{.2in}

\begin{itemize}
  %
\item{Use Reference Layers such as Units, Sections, Quarter Sections, and Parcels.

  \begin{figure}[H]
  \centering
  \includegraphics[width=.7\textwidth]{selectReferenceLayers.png}
  \vspace{-.1in}

  \caption{Select Reference Layers}
  \end{figure}
}
  %
\clearpage

\item{Use the Measuring Tool

  \begin{figure}[H]
  \centering
      \includegraphics[width=1\textwidth]{measuringToolInToolbar.png}
  \vspace{-.2in}

  \caption{Measuring Tool}
  \end{figure}
}
  %
\vspace{.3in}

\item{Search by Parcel Number \small(Search Layers Plugin)\\
\begin{figure}[H]
\centering
    \includegraphics[width=1\textwidth]{searchLayersIcon.PNG}
\caption{Search Layer Icon}
\end{figure}
}
  %
\item{Draw COGO lines \small(Step 4)}
  %
\end{itemize}
  %
\clearpage

\stepcounter{stepCount}

\subparagraph*{{\LARGE Step \thestepCount:} Draw a Line With Azimuth and Distance}

\begin{figure}[H]
  %
\centering
    \includegraphics[width=1\textwidth]{DescriptionFromDeedHighlighted.PNG}
\vspace{-.1in}

\caption{Description From Deed}
\end{figure}
  %
On the \textbf{Drawing Tab:}
\begin{itemize}
  %
\item Azimuth (bearing): Enter Bearing in format: \emph{N 88 32 05 W}
  %
\item Offset: Set to \emph{0}
  %
\item Zenith: Set to \emph{90}
  %
\item Distance: Enter Feet Distance in numbers only \emph{1338.44}
  %
\end{itemize}
  %
\begin{figure}[H]
  %
\centering
    \includegraphics[width=1\textwidth]{enteringBounds.png}
\vspace{-.1in}

\caption{Entering Bounds}
\end{figure}

{\LARGE Push \fbox {Add to Bottom}}

\clearpage


\subparagraph*{Line is added to the list}

\begin{figure}[H]
  %
\centering
    \includegraphics[width=1\textwidth]{lineAdded.png}
\vspace{-.1in}

\caption{Line Added}
\end{figure}


\subparagraph*{Add as many bounds as you can from the description}

\begin{figure}[H]
  %
\centering
    \includegraphics[width=1\textwidth]{threeLinesAdded.png}
\vspace{-.1in}

\caption{Three Lines Added}
\end{figure}

\clearpage

\subparagraph*{Choose A Point to Start Drawing From\\}
Push the \textbf{\fbox{From Map}} button.\\

\textasteriskcentered{\scriptsize Decide which layer to reference for a starting point.}
  %
  
\noindent Align cursor with desired starting point and click.
  %
\begin{figure}[H]
  %
\centering
    \includegraphics[width=1\textwidth]{fromMap.png}
\vspace{-.1in}

\caption{From Map}
\end{figure}

\clearpage

\subparagraph*{Draw the Segments So Far}
\vspace{.2in}

\begin{itemize}
  %
\item Push \textbf{\fbox{Draw}}
  %
\item Enter Attributes for the polyline to be created
  %
\item Press \textbf{\fbox{OK}}
  %
\end{itemize}

\begin{figure}[H]
  %
\centering
    \includegraphics[width=1\textwidth]{enterAttributes.PNG}
\vspace{-.1in}

\caption{Enter Attributes}
\end{figure}

\clearpage

\subparagraph*{Use the sketch to identify the parcel\\}

In this case, turn on ortho photo to verify the remaining bounds.

\begin{figure}[H]
  %
\centering
    \includegraphics[width=1\textwidth]{verifyRemainingBounds.PNG}
\vspace{-.1in}

\caption{Verify Remaining Bounds}
\end{figure}

\subparagraph*{(optionally) Save Input for Later Use\\}

If you want to save the segments for later use, press \textbf{\fbox{Export}}.\\

\noindent Name it and select a \textbf{save} location.

\begin{figure}[H]
  %
\centering
    \includegraphics[width=1\textwidth]{saveSegmentList.png}
\vspace{-.2in}

\caption{Save Segment List}
\end{figure}

\clearpage

\subparagraph*{Verify Attributes\\}

Right click on \emph{\fbox{Traverse1}} in the Layers Panel\\

\noindent and select \textbf{\fbox{open attribute table}}.\\

\noindent {\footnotesize The attributes you entered should be in the table.}

\begin{figure}[H]
  %
\centering
    \includegraphics[width=1\textwidth]{segmentsInTable.png}
\vspace{-.1in}

\caption{Segments In Table}
\end{figure}
  %
\clearpage
  %
\end{document}

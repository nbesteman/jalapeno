\documentclass[class=article, crop=false, titlepage, twoside, multi={itemize, figure, verbatim}, float=false]{standalone}



%%%%%%%%%%%%%%%%%%%%%%%%%%%%%%%%%%%%%%%%%%%%%%%%%%%%%%%%%%%%%
% Tried and True Preamble Section
\usepackage{import} % Required for importing other .tex docs.  (import uses everything bw Begin and End Doc)
\usepackage{float} % Required for specifying the exact location of a figure or table
\usepackage{wrapfig} % Provides environment to wrap figures with text
\usepackage{graphicx} % Required for including images
\usepackage{cite} % improves citation handling
%\usepackage[toc,title,page]{appendix}
\usepackage{pdfpages} % enables loading a pdf into the doc
\usepackage{makeidx} % Enables index features
\usepackage{glossaries} % must be after hyperref
\usepackage{blindtext} % enables ipsum loerem "dummy" text blocks
\usepackage{enumitem} % enables control over enumerate, itemize, and description
\usepackage[dvipsnames]{xcolor}%black, blue, brown, cyan, darkgray, gray, green, lightgray, lime, magenta, olive, orange, pink, purple, red, teal, violet, white, yellow.
%
\definecolor{HeaderBlueA1}{RGB}{10, 91, 163}
\definecolor{HeaderBlueA2}{RGB}{101, 98, 174}
\definecolor{HeaderBlueA3}{RGB}{84, 125, 161}
\definecolor{HeaderBlueB}{RGB}{53, 98, 138}
\definecolor{HeaderBlueC}{RGB}{31,78,121}  % HeaderBlue1
\definecolor{HeaderBlueD}{RGB}{16, 60, 99}
\definecolor{HeaderBlueE}{RGB}{4, 40, 72 }

\definecolor{HeaderOrangeA}{RGB}{249, 161, 121}
\definecolor{HeaderOrangeB}{RGB}{214, 116, 72}
\definecolor{HeaderOrangeC}{RGB}{187, 84, 37}
\definecolor{HeaderOrangeD}{RGB}{152, 57, 14}
\definecolor{HeaderOrangeE}{RGB}{111, 35, 0}

\definecolor{HeaderGoldA}{RGB}{249, 218, 121}
\definecolor{HeaderGoldB}{RGB}{214, 180, 72}
\definecolor{HeaderGoldC}{RGB}{187, 151, 37}
\definecolor{HeaderGoldD}{RGB}{152, 119, 14}
\definecolor{HeaderGoldE}{RGB}{111, 84, 0}

% Color triad based on HeaderBlue1
% http://paletton.com/
% Blues:
% 84, 125, 161 lightest blue
% 53, 98, 138 light blue
% 31,78,121 HeaderBlue1
% 16, 60, 99 darker blue
% 4, 40, 72 darkest blue
% Oranges:
% 249, 161, 121 lightest orange
% 214, 116, 72 light orange
% 187, 84, 37 main orange
% 152, 57, 14 darker orange
% 111, 35, 0 dark orange
% Golds:
% 249, 218, 121 lightest goldenrod
% 214, 180, 72 light goldenrod
% 187, 151, 37 goldenrod
% 152, 119, 14 dark goldenrod
% 111, 84, 0 darkest goldenrod

\definecolor{HyperlinkBlue1}{RGB}{64,172,209}
\definecolor{graphicOrange}{RGB}{255,153,51}
% Header1 is Arial(body)(Caps) 14pt
% Header2 is Arial(body) 12pt
% Body is arial 9pt
\usepackage{marginnote}
%\usepackage{geometry}
\usepackage[top=1.5in, bottom=1.5in, outer=2.5in, inner=1.25in, heightrounded, marginparwidth=1.5in, marginparsep=.25in]{geometry} % Sets page geometry
\usepackage{fancyhdr} % Adds control of headers and footers
%%%%  Testing
\usepackage{pifont}
\usepackage{anyfontsize}
\usepackage{multicol}
%%%%%%%%%%%%%


%\newcommand{\localtextbulletone}{\textcolor{HeaderOrangeC}{\raisebox{.45ex}{\rule{.6ex}{.6ex}}}}

% ding characters specified in symbols-a4 pg12
\newcommand{\localtextbulletone}{\textcolor{HeaderOrangeC}{\ding{227}}}
\renewcommand{\labelitemi}{\localtextbulletone}

\newcommand{\HRule}{\rule{\linewidth}{0.5mm}} % Command to make horizontal graphic lines
\newcommand{\marginText}{\fontfamily{cmr}\fontseries{b}\color{red}\fontsize{9}{12}\selectfont}
\newcommand{\helv}{\fontfamily{phv}\fontseries{b}\fontsize{9}{11}\selectfont}
% font has encoding, family,  series{b}(bold), shape, size{(size)}{(baselineskip)}
%       % https://www.latex-project.org/help/documentation/fntguide.pdf
%       % https://www.overleaf.com/learn/latex/Font_typefaces
%
\graphicspath{{img/}{GIS_ChampionSection/img/}{awardsChapter/GIS_ChampionSection/img/}{brandPart/awardsChapter/GIS_ChampionSection/img/}{img/}{pairedProgSection/img/}{methodChapter/pairedProgSection/img/}{methodPart/methodChapter/pairedProgSection/img/}{documentationSection/img/}{methodChapter/documentationSection/img/}{methodPart/methodChapter/documentationSection/img/}{docStorageOrgSection/img/}{methodChapter/docStorageOrgSection/img/}{methodPart/methodChapter/docStorageOrgSection/img/}{QGisSection/img/}{toolsChapter/QGisSection/img/}{servicePart/toolsChapter/QGisSection/img/}{ESRISection/img/}{toolChapter/ESRISection/img/}{servicePart/toolChapter/ESRISection/img/}{../../../../source/}{../../source/}{servicePart/applicationsChapter/treasurerSection/img/}{servicePart/toolsChapter/gisAdminSection/img/}{gisAdminSection/img/}{servicePart/toolsChapter/bsaSupportSection/img/}{servicePart/toolsChapter/coreDataSection/img/}{../img}}
%
%%%%%%%%%%%%%%%%%%%%%%%%%%%%%%%%%%%%%%%%%%%%%%%%%%%%%%%%%%%%%
%\usepackage{fancyhdr} % Adds control of headers and footers
%
%     Code Key: E : Even page, O : Odd page, L Left field,  C : Center field,
%               R : Right field, H : Header, F : Footer
%    \leftmark : Contains the Left argument of the last \markboth on the pages
%    \rightmark : Contains the Right Argument of the first \markboth of the page OR:
%                 the first /markright of the page
%
\renewcommand{\chaptermark}[1] % Redefinition of how chapter mark is displayed
{\markboth{\MakeUppercase{\thechapter.\ #1}}{}}
%
\renewcommand{\subsectionmark}[1] % Redefinition of how subsection mark is displayed
{\markright{\MakeUppercase{\thesubsection.\ #1}}}
%
\renewcommand{\headrulewidth}{0.5pt} % Line after header
\renewcommand{\footrulewidth}{0.5pt} % line before footer
%
\fancyhf{} % clear all header and footer fields
%
\fancyhead[LE,RO]{\helv \thepage} % Left Even and Right Odd headers
%                                 % \thepage macro displays the current page number
\fancyhead[LO]{\helv \leftmark} % Left Odd headers are set to /markboth
%                               % (chapters from \renewcommand{\chaptermark}[1]... )
\fancyhead[RE]{\helv \rightmark} % Right Even Headers
%                                % (subsection from \renewcommand{\subsectionmark}[1]...)




%\fancyheadoffset[RE,LO,RO,LE]{.5\textwidth}
%

% Alternate fancy header options
%\renewcommand{\sectionmark}[1] % Redefinition of how section mark is displayed
%{\markright{\MakeUppercase{\thesection.\ #1}}}
%
% Alternate fancy footer options
%\fancyfoot[C]{\textbf{\thepage}} % Page number on bottom center
%\fancyfoot[EC,OC]{\helv \thesection} % Section number bottom center

%%%%%%%%%%%%%%%%%%%%%%%%%%%%%%%%%%%%%%%%%%%%%%%%%%%%%%%%%%
% Test area

%\setlength\parindent{0pt} % eliminates indents
%\tolerance=1500 % Number affects overfull and Underful box warnings
%\usepackage{caption}

%\setlist[description]{leftmargin=\parindent,labelindent=\parindent}
\newcommand{\titleGeometry}{\geometry{top=1in,bottom=1in,inner=1.5in,outer=1.5in}} % Sets geometry for first part}

\usepackage[pdftex,breaklinks,colorlinks=true,linkcolor=black,citecolor=blue,urlcolor=red,linktocpage=false,pagebackref=true,filecolor=magenta]{hyperref}% http://www.tug.org/applications/hyperref/manual.html#x1-100003.6
\hypersetup{pdfauthor={Allegan County GIS Services},% Page 248 in Latex Beinners Guide
pdftitle={What We Do},
pdfsubject={Documentation of everything},
pdfkeywords={documentation,gis}}


\def\titlename{The hyperref Package\\ \medskip\large \LaTeX{} Packages }

\title{\HRule % Horizontal Line added
\\[.4cm] % space
\protect\begin{figure}[H] % included image
\protect\begin{center}	% centered horizontally
\includegraphics[scale=.45]{GIS_Logo_better.jpg}
\protect\end{center}
\protect\end{figure}
\Huge \bfseries \titlename \\ % Title text
\HRule \\[.4cm] % Horizontal Line added
\author{\Large Allegan County GIS \\ \Large www.allegancounty.org/gis} % defines author
}  % inputs common title

\begin{document}% Document Begins

\ifstandalone
\maketitle % creates title page
\clearpage
\tableofcontents % creates TOC
\clearpage
\fi

\subsection[hyperref Package]{\LARGE hyperref Package}

\subsubsection[usepackage]{\Large usepackage}

Add the \textit{hyperref package} to the preamble\begin{large}\textbf{ last}\end{large}.	

\begin{verbatim}\usepackage[options]{hyperref}\end{verbatim}

\subsubsection[Simple Use]{\Large Simple Use}
\begin{verbatim}Use \href{URL}{DESCRIPTION} to add a link with description

\href{https://www.latex-tutorial.com}{Website with tutorials}
produces:
\end{verbatim}
\href{https://www.latex-tutorial.com}{Website with tutorials}

\subsubsection[Options]{\Large Options}
Add optional arguments to the usepackage line:\\
Useful options:\begin{itemize}
\item \begin{large}\textbf{pdftex}\end{large}\\enables other options like breaklines
\item \begin{large}\textbf{breaklinks}\end{large}\\allow links to be broken across several lines\\eg. \url{https://lists.gnu.org/archive/html/emacs-orgmode/2013-06/msg00776.html}
\item \begin{large}\textbf{colorlinks}\end{large}\\Colors the text of links and anchors.(default is false)
\item \begin{large}\textbf{linkcolor}\end{large}\\Color for normal internal links(default is red).
\item \begin{large}\textbf{anchorcolor}\end{large}\\Color for anchor text.
\item \begin{large}\textbf{citecolor}\end{large}\\Color for bibliographic citations in text.
\item \begin{large}\textbf{urlcolor}\end{large}\\Color for linked URLs

\end{itemize}

\subsubsection[Use with Options]{\Large Use with options}
\begin{verbatim}
\usepackage[breaklinks,colorlinks,citecolor=blue,
urlcolor=green]{hyperref}
\end{verbatim}

\subsubsection[Commands]{\Large Commands}
\begin{verbatim}
\href{URL}{text} — Makes text a link to URL.
\end{verbatim}

\begin{Large}To put a file path in text:\end{Large}\\
eg:\\
%\href{../../../../../documentation/packageDocs/hyperref2017.pdf}{Official hyperref package documentation}
\ifstandalone
This link uses path relative to subsection document\\
\href{../../../../../documentation/packageDocs/hyperref2017.pdf}{Official hyperref package documentation}\\
The second link will only work from the main document
\fi

\href{../../documentation/packageDocs/hyperref2017.pdf}{Official hyperref package documentation}\\
%\href{C:/AC/jalapeno/documentation/packageDocs/hyperref2017.pdf}{Official hyperref package documentation}
\begin{tiny}(documentation Pt.4 pg.15)\end{tiny}
\begin{verbatim}
\href[options}{URL}{text}
\end{verbatim}

Options:\begin{itemize}
\item absolute
\begin{footnotesize}
\begin{verbatim}
\href{C:/AC/jalapeno/documentation/packageDocs/hyperref2017.pdf}
    {Official hyperref doc}
\end{verbatim}
\end{footnotesize}

\item relative \textbf{Note: relative path must be from final pdf location}
\begin{footnotesize}
\begin{verbatim}
\href{../../../../../documentation/packageDocs/hyperref2017.pdf}
    {Official hyperref package doc}
\end{verbatim}
*This path works from main document
\begin{verbatim}

\href{../../documentation/packageDocs/hyperref2017.pdf}
    {Official hyperref package documentation}
\end{verbatim}
*This path works from subsection document

\end{footnotesize}

\end{itemize}


\begin{verbatim}
\hyperref[label]{text} 
    Makes text a link to where \ref{label} would point.
  
\hypertarget{name}{text}
    Sets an anchor on text with the label name.
     
\hyperlink{name}{text} 
    Makes text a link that takes you to the anchor labeled name.     
    *Pair with \hypertarget.
 
\phantomsection
    Used in conjunction with

\addcontentsline 
    to make the correct link in the Table of Contents.
\end{verbatim}				

\end{document}

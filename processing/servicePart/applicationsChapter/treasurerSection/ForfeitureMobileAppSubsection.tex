\documentclass[class=article , crop=false, titlepage, twoside, multi={itemize, figure, verbatim}, float=false]{standalone}



%%%%%%%%%%%%%%%%%%%%%%%%%%%%%%%%%%%%%%%%%%%%%%%%%%%%%%%%%%%%%
% Tried and True Preamble Section
\usepackage{import} % Required for importing other .tex docs.  (import uses everything bw Begin and End Doc)
\usepackage{float} % Required for specifying the exact location of a figure or table
\usepackage{wrapfig} % Provides environment to wrap figures with text
\usepackage{graphicx} % Required for including images
\usepackage{cite} % improves citation handling
%\usepackage[toc,title,page]{appendix}
\usepackage{pdfpages} % enables loading a pdf into the doc
\usepackage{makeidx} % Enables index features
\usepackage{glossaries} % must be after hyperref
\usepackage{blindtext} % enables ipsum loerem "dummy" text blocks
\usepackage{enumitem} % enables control over enumerate, itemize, and description
\usepackage[dvipsnames]{xcolor}%black, blue, brown, cyan, darkgray, gray, green, lightgray, lime, magenta, olive, orange, pink, purple, red, teal, violet, white, yellow.
%
\definecolor{HeaderBlueA1}{RGB}{10, 91, 163}
\definecolor{HeaderBlueA2}{RGB}{101, 98, 174}
\definecolor{HeaderBlueA3}{RGB}{84, 125, 161}
\definecolor{HeaderBlueB}{RGB}{53, 98, 138}
\definecolor{HeaderBlueC}{RGB}{31,78,121}  % HeaderBlue1
\definecolor{HeaderBlueD}{RGB}{16, 60, 99}
\definecolor{HeaderBlueE}{RGB}{4, 40, 72 }

\definecolor{HeaderOrangeA}{RGB}{249, 161, 121}
\definecolor{HeaderOrangeB}{RGB}{214, 116, 72}
\definecolor{HeaderOrangeC}{RGB}{187, 84, 37}
\definecolor{HeaderOrangeD}{RGB}{152, 57, 14}
\definecolor{HeaderOrangeE}{RGB}{111, 35, 0}

\definecolor{HeaderGoldA}{RGB}{249, 218, 121}
\definecolor{HeaderGoldB}{RGB}{214, 180, 72}
\definecolor{HeaderGoldC}{RGB}{187, 151, 37}
\definecolor{HeaderGoldD}{RGB}{152, 119, 14}
\definecolor{HeaderGoldE}{RGB}{111, 84, 0}

% Color triad based on HeaderBlue1
% http://paletton.com/
% Blues:
% 84, 125, 161 lightest blue
% 53, 98, 138 light blue
% 31,78,121 HeaderBlue1
% 16, 60, 99 darker blue
% 4, 40, 72 darkest blue
% Oranges:
% 249, 161, 121 lightest orange
% 214, 116, 72 light orange
% 187, 84, 37 main orange
% 152, 57, 14 darker orange
% 111, 35, 0 dark orange
% Golds:
% 249, 218, 121 lightest goldenrod
% 214, 180, 72 light goldenrod
% 187, 151, 37 goldenrod
% 152, 119, 14 dark goldenrod
% 111, 84, 0 darkest goldenrod

\definecolor{HyperlinkBlue1}{RGB}{64,172,209}
\definecolor{graphicOrange}{RGB}{255,153,51}
% Header1 is Arial(body)(Caps) 14pt
% Header2 is Arial(body) 12pt
% Body is arial 9pt
\usepackage{marginnote}
%\usepackage{geometry}
\usepackage[top=1.5in, bottom=1.5in, outer=2.5in, inner=1.25in, heightrounded, marginparwidth=1.5in, marginparsep=.25in]{geometry} % Sets page geometry
\usepackage{fancyhdr} % Adds control of headers and footers
%%%%  Testing
\usepackage{pifont}
\usepackage{anyfontsize}
\usepackage{multicol}
%%%%%%%%%%%%%


%\newcommand{\localtextbulletone}{\textcolor{HeaderOrangeC}{\raisebox{.45ex}{\rule{.6ex}{.6ex}}}}

% ding characters specified in symbols-a4 pg12
\newcommand{\localtextbulletone}{\textcolor{HeaderOrangeC}{\ding{227}}}
\renewcommand{\labelitemi}{\localtextbulletone}

\newcommand{\HRule}{\rule{\linewidth}{0.5mm}} % Command to make horizontal graphic lines
\newcommand{\marginText}{\fontfamily{cmr}\fontseries{b}\color{red}\fontsize{9}{12}\selectfont}
\newcommand{\helv}{\fontfamily{phv}\fontseries{b}\fontsize{9}{11}\selectfont}
% font has encoding, family,  series{b}(bold), shape, size{(size)}{(baselineskip)}
%       % https://www.latex-project.org/help/documentation/fntguide.pdf
%       % https://www.overleaf.com/learn/latex/Font_typefaces
%
\graphicspath{{img/}{GIS_ChampionSection/img/}{awardsChapter/GIS_ChampionSection/img/}{brandPart/awardsChapter/GIS_ChampionSection/img/}{img/}{pairedProgSection/img/}{methodChapter/pairedProgSection/img/}{methodPart/methodChapter/pairedProgSection/img/}{documentationSection/img/}{methodChapter/documentationSection/img/}{methodPart/methodChapter/documentationSection/img/}{docStorageOrgSection/img/}{methodChapter/docStorageOrgSection/img/}{methodPart/methodChapter/docStorageOrgSection/img/}{QGisSection/img/}{toolsChapter/QGisSection/img/}{servicePart/toolsChapter/QGisSection/img/}{ESRISection/img/}{toolChapter/ESRISection/img/}{servicePart/toolChapter/ESRISection/img/}{../../../../source/}{../../source/}{servicePart/applicationsChapter/treasurerSection/img/}{servicePart/toolsChapter/gisAdminSection/img/}{gisAdminSection/img/}{servicePart/toolsChapter/bsaSupportSection/img/}{servicePart/toolsChapter/coreDataSection/img/}{../img}}
%
%%%%%%%%%%%%%%%%%%%%%%%%%%%%%%%%%%%%%%%%%%%%%%%%%%%%%%%%%%%%%
%\usepackage{fancyhdr} % Adds control of headers and footers
%
%     Code Key: E : Even page, O : Odd page, L Left field,  C : Center field,
%               R : Right field, H : Header, F : Footer
%    \leftmark : Contains the Left argument of the last \markboth on the pages
%    \rightmark : Contains the Right Argument of the first \markboth of the page OR:
%                 the first /markright of the page
%
\renewcommand{\chaptermark}[1] % Redefinition of how chapter mark is displayed
{\markboth{\MakeUppercase{\thechapter.\ #1}}{}}
%
\renewcommand{\subsectionmark}[1] % Redefinition of how subsection mark is displayed
{\markright{\MakeUppercase{\thesubsection.\ #1}}}
%
\renewcommand{\headrulewidth}{0.5pt} % Line after header
\renewcommand{\footrulewidth}{0.5pt} % line before footer
%
\fancyhf{} % clear all header and footer fields
%
\fancyhead[LE,RO]{\helv \thepage} % Left Even and Right Odd headers
%                                 % \thepage macro displays the current page number
\fancyhead[LO]{\helv \leftmark} % Left Odd headers are set to /markboth
%                               % (chapters from \renewcommand{\chaptermark}[1]... )
\fancyhead[RE]{\helv \rightmark} % Right Even Headers
%                                % (subsection from \renewcommand{\subsectionmark}[1]...)




%\fancyheadoffset[RE,LO,RO,LE]{.5\textwidth}
%

% Alternate fancy header options
%\renewcommand{\sectionmark}[1] % Redefinition of how section mark is displayed
%{\markright{\MakeUppercase{\thesection.\ #1}}}
%
% Alternate fancy footer options
%\fancyfoot[C]{\textbf{\thepage}} % Page number on bottom center
%\fancyfoot[EC,OC]{\helv \thesection} % Section number bottom center

%%%%%%%%%%%%%%%%%%%%%%%%%%%%%%%%%%%%%%%%%%%%%%%%%%%%%%%%%%
% Test area

%\setlength\parindent{0pt} % eliminates indents
%\tolerance=1500 % Number affects overfull and Underful box warnings
%\usepackage{caption}

%\setlist[description]{leftmargin=\parindent,labelindent=\parindent}
\newcommand{\titleGeometry}{\geometry{top=1in,bottom=1in,inner=1.5in,outer=1.5in}} % Sets geometry for first part}

\usepackage[pdftex,breaklinks,colorlinks=true,linkcolor=black,citecolor=blue,urlcolor=red,linktocpage=false,pagebackref=true,filecolor=magenta]{hyperref}% http://www.tug.org/applications/hyperref/manual.html#x1-100003.6
\hypersetup{pdfauthor={Allegan County GIS Services},% Page 248 in Latex Beinners Guide
pdftitle={What We Do},
pdfsubject={Documentation of everything},
pdfkeywords={documentation,gis}}


\def\titlename{Forfeiture Data Collection\\ \medskip\large Mobile App with Collector for ArcGIS}

\title{\HRule % Horizontal Line added
\\[.4cm] % space
\protect\begin{figure}[H] % included image
\protect\begin{center}	% centered horizontally
\includegraphics[scale=.45]{GIS_Logo_better.jpg}
\protect\end{center}
\protect\end{figure}
\Huge \bfseries \titlename \\ % Title text
\HRule \\[.4cm] % Horizontal Line added
\author{\Large Allegan County GIS \\ \Large www.allegancounty.org/gis} % defines author
}  % inputs common title
\setcounter{tocdepth}{5}  % subparagraph and down
\begin{document}% document begins

\ifstandalone
\maketitle % creates title page
\clearpage
\tableofcontents % creates TOC
\clearpage
\fi

\subsection{Forfeiture Data Collection}

\subsubsection{Problem and Analysis}

\paragraph{Background}
Treasurer department has an annual responsibility to properly document the tax forfeiture process.  The LIS Department built an application in MS Access and MapInfo that consumed a daily export from BSA and was deployed to the field on a laptop.  A digital camera was used for site photos and later imported into the laptop.

\paragraph{Statement of Problem}
Current Tax Forfeiture workflow is built on MapInfo software which has been replaced by ESRI software.  The Forfeiture data collection application must be recreated in the ESRI framework.

\paragraph{Analysis}
Tax Forfeiture Application will facilitate:

\begin{itemize} %1

\item Mobile data collection on handheld device via Collector for ArcGIS configured with Allegan County GIS Portal  (\textbf{device app})

\begin{itemize} %2

\item Device app will:

\begin{itemize} %3

\item Synchronize with data in the office (online)
\item Navigate to forfeiture sites (offline)
\item Collect data and photos of forfeiture sites (offline)
\item Synchronize the collected data with data in the office (online)
\end{itemize} %3

\end{itemize} %2

\item Daily form production and printing for each site visited with required data and images.

\end{itemize} %1

\clearpage
\subsubsection{Design}
\paragraph{Overview}This Application utilizes Treasurer Department data to document the forfeiture process.  An enterprise GIS deployment enables offline data collection by up to two users.
\begin{figure}
\centering
    \includegraphics[width=1\textwidth]{DesignFlowChart}
\caption{Project Design}
\end{figure}
\subparagraph*{}There are three stages to daily workflow: Preprocessing, Field Collection, and Postprocessing.  Forfeiture Parcels, is a map feature class that is processed in the office via the network and remotely via the internet.
\clearpage
\subparagraph{Workflow Summary}
\begin{description}
\item [Preprocessing] The data is updated to match the Treasures data in BSAforfeiture.net and synchronized to two android mobile devices.
\item [Field data collection] The two mobile devices are used to collect info required, one for all the attributes, the other for photos.
\item[Postprocessing] The mobile devices are syncronized back to the network data and a form is exported for each site visited that day.
\end{description}


\paragraph{Technologies Used}
\subparagraph{BSA Data}Details of parcels in the forfeiture process are managed in BSA Delinquent Tax.net.  The Treasurer office does a BSA export of the parcels in need of a site visit in the preprocessing.

\subparagraph{ArcGIS Desktop}Tools are designed to preprocess and postprocess forfeiture parcel data for fieldwork.  The user will execute a preprocess script tool that prepares the data for field deployment.  After fieldwork, a post process script tool syncronizes data from the fieldwork with the live data on the Allegan County network. 

\subparagraph{ArcGIS Collector}A free mobile application developed and tested on Android is deployed to the field for data collection.  The application is configured to work offline(without an internet or cellular connection) by syncronizing before and after fieldwork.

\subparagraph{ArcGIS Portal Webmaps and Apps}Live data from a publishing enterprise geodatabase(ACPub), running on SQL Server database server (acintsql01) is provided through a feature service (REST service)  named TaxReversionParcels.  A webmap called the Forfeiture Field Map consumes the TaxReversionParcels feature service, exposing the data to editing.  The Forfeiture Field Map is configured to work in the ArcGIS Collector App.  The app downloads the webmap, allowing the user to collect the necessary information on each forfeiture parcel in the field disconnected and uploads the changes when reconnected. 

\paragraph{Field Data Collection}

Three parts of the daily routine:
\begin{enumerate}
\item Pre-processing (in the office):

\begin{itemize}
\item Export current forfeiture list from BSA
\item Update webmap layers with results from BSA export
\item Synchronize from webmap layers to field collection devices \textbf{(device app)}
\end{itemize}

\item Field data collection with device app:

\begin{itemize}
\item Navigation to forfeiture sites is aided by users location shown in map
\item A Checklist of data points about the site
\item Attach photos to the site
\item Save results for synchronization in post-processing
\end{itemize}

\item Post-processing (in the office)

\begin{itemize}
\item Synchronize data and images collected in device app to webmap layers

\end{itemize}
\end{enumerate}

\paragraph{Data Details}
\subparagraph{Location of Production Data}
\begin{wrapfigure}{r}{0.5\textwidth}
\centering
\includegraphics[width=.45\textwidth]{ForfParcelsCatalog3}
\caption{Live Data Location Screnshot}
\end{wrapfigure}
The data is located in ACPUB.
\clearpage

\subparagraph{ForfeitureParcels Feature Class}

\paragraph{Collector for ArcGIS}

\clearpage
\paragraph{Webmap Details}

\clearpage
\subsubsection{Hard Copy Record}


\clearpage
\subsubsection{User Manual}

\paragraph{Admin Tasks}

\subparagraph{Setup Users in ArcGIS}Users that will run Pre and Post processing scripts must be created and given priviliges on ACPub Treasurer Feature Data Set.

\subparagraph{Setup Users in Portal for ArcGIS}Users that will use the Collector for ArcGIS must have profiles added to and managed in the Allegan County GIS Portal site.


\paragraph{Collector Setup Details}

\subparagraph{Install Collector for ArcGIS}
\begin{itemize}
\item Available from the Google Play Store
\end{itemize}
\subparagraph{Configure Collector}
\begin{itemize}
\item Connect to Allegan County GIS

\begin{itemize}
\item Choose or add the connection:

\begin{verbatim}
https://gis.allegancounty.org/portal_webadaptor
\end{verbatim}

\begin{figure}
\centering
\includegraphics[width=.75\textwidth]{CollectorConnection}
\caption{Collector Connection Screenshot}
\end{figure}

\item Username is JMorris or CAndress
\item Password: (enter password)
\end{itemize}

\item Find the map Forfeiture Field Map under Treasury Services
\item Download the field map
\item Select area needed and detail needed and download the webmap

\end{itemize}


\paragraph{Daily Preprocessing Routine}

\subparagraph{Execute Preprocessing Script}A tool in ArcGIS that:

\begin{itemize}

\item Exports current forfeiture list from BSA
\item Updates webmap layers with results from BSA export

%Insert screenshot of Preprocesing script in Arc here

\end{itemize}

\subparagraph{Synchronize Webmap}In Collector for ArcGIS, push the sync button on the Forfeiture Field Map 
\newpage
\paragraph{Forfeiture Data Collection}
\subparagraph{Forfeiture Parcels Data Details}
Attributes are of four entry types:\begin{itemize}
\item prefilled
\item autofill
\item dropdown
\item text box \end{itemize}
In the Forfeiture Field Map, for each site visited, select the desired parcel, push the edit button and collect attributes.  If the boxes are autofill, select from dropdown or typed.\bigskip 
\begin{table}
\centering
\begin{tabular}{|l|c|r|}
\hline
\multicolumn{3}{|c|}{Attribute List} \\
\hline
Field Name&Entry Type&Note\\ \hline
Property Number&Prefilled&NA\\
Inspection Date&{\scriptsize Autofill or Dropdown}&NA\\
Inspector&Dropdown&NA\\
Class&Prefilled&NA\\
Acres&Prefilled&NA\\
Address&Prefilled&NA\\
Status&Dropdown&NA\\
Status Notes&Open entry&254 Char limit\\
Road Frontage&Dropdown&Yes or No\\
Access via&Open entry&30 Char Limit\\
Agent&Open entry&30 Char Limit\\
Agent Contact&Open entry&30 Char Limit\\
Property in use&Dropdown&Yes or\\
Use Notes&Open entry&254 Char limit\\
Property Maintained&Dropdown&Yes or No\\
Notes&Dropdown&254 Char limit\\
Prop Contam&Dropdown&Yes or No\\
Notes&Open entry&254 Char limit\\
Adj Prop Contam&Dropdown&NA\\
Notes&Open entry&254 Char limit\\
Property for sale&Dropdown&Yes or No\\
Posted&Prefilled&in Pre and Postproc\\
InList&Prefilled&in Preproc\\
PostedInList&Prefilled&in Preproc\\
Print Today&Dropdown&Yes or No\\ \hline
\end{tabular}
\caption{Dataset Details}
\end{table}

\clearpage
\subparagraph{Device 1 Field Operation}
\subparagraph*{}In the Forfeiture Field Map, for each site visited, select the desired parcel, push the edit button and then edit attributes.

\begin{wrapfigure}{r}{0.5\textwidth}
\centering
\includegraphics[width=.3\textwidth]{editsInterface}
\caption {Device 1 Data Entry}
\end{wrapfigure}
\vspace{1in}

This figure shows the data collection interface.  Device one will be used to add data to all of the boxes.  Touch the boxes to enter text or select a dropdown.

\clearpage
\subparagraph{Device 2 Field Operation} 
\subparagraph*{}In the Forfeiture Field Map, for each site visited, select the desired parcel, push the edit button and then the add attachment button.  Select photo and take a photo.

\begin{wrapfigure}{l}{0.5\textwidth}
\centering
\includegraphics[width=.3\textwidth]{photosInterface}
\caption {Device 2 Data Entry}
\end{wrapfigure}
\vspace{1in}

This figure shows the data collection interface.  Device two will be used to add photos to a parcel.

\clearpage
\paragraph{Daily Postprocessing Routine}Back at the office
\subparagraph{Synchronize Webmap}In Collector for ArcGIS, push the sync button on the Forfeiture Field Map
\subparagraph{Execute Postprocessing Script}A tool in ArcGIS that:

\begin{itemize}
\item Reconciles geodatabase versions
\item Generates forms for each site visited


%Insert screenshot of Postprocesing script in Arc here

\end{itemize}

%\begin{description}
%\item [Sync Edits] \blindtext
%\item [reconcile Versions] \blindtext
%\item[Print forms for site visits] \blindtext
%\item[Update BSA] \blindtext
%\end{description}

\clearpage
\subsubsection{Software}
\paragraph{ESRI Licensed Products}
\subparagraph{ArcDesktop}Users of this application need a license to ArcGIS Standard level.

\subparagraph{Enterprise ArcGIS Deployment}This app uses ArcGIS Server and ArcGIS Portal.

\subparagraph{Collector for ArcGIS}Developed and tested on Android(7.0).  Collector is available at the Google Play Store.

\end{document}
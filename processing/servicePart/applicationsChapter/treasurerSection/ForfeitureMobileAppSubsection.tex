\documentclass[class=article , crop=false, titlepage, twoside, multi={itemize, figure, verbatim}, float=false]{standalone}



%%%%%%%%%%%%%%%%%%%%%%%%%%%%%%%%%%%%%%%%%%%%%%%%%%%%%%%%%%%%%
% Tried and True Preamble Section
\usepackage{import} % Required for importing other .tex docs.  (import uses everything bw Begin and End Doc)
\usepackage{float} % Required for specifying the exact location of a figure or table
\usepackage{wrapfig} % Provides environment to wrap figures with text
\usepackage{graphicx} % Required for including images
\usepackage{cite} % improves citation handling
%\usepackage[toc,title,page]{appendix}
\usepackage{pdfpages} % enables loading a pdf into the doc
\usepackage{makeidx} % Enables index features
\usepackage{glossaries} % must be after hyperref
\usepackage{blindtext} % enables ipsum loerem "dummy" text blocks
\usepackage{enumitem} % enables control over enumerate, itemize, and description
\usepackage[dvipsnames]{xcolor}%black, blue, brown, cyan, darkgray, gray, green, lightgray, lime, magenta, olive, orange, pink, purple, red, teal, violet, white, yellow.
%
\definecolor{HeaderBlueA1}{RGB}{10, 91, 163}
\definecolor{HeaderBlueA2}{RGB}{101, 98, 174}
\definecolor{HeaderBlueA3}{RGB}{84, 125, 161}
\definecolor{HeaderBlueB}{RGB}{53, 98, 138}
\definecolor{HeaderBlueC}{RGB}{31,78,121}  % HeaderBlue1
\definecolor{HeaderBlueD}{RGB}{16, 60, 99}
\definecolor{HeaderBlueE}{RGB}{4, 40, 72 }

\definecolor{HeaderOrangeA}{RGB}{249, 161, 121}
\definecolor{HeaderOrangeB}{RGB}{214, 116, 72}
\definecolor{HeaderOrangeC}{RGB}{187, 84, 37}
\definecolor{HeaderOrangeD}{RGB}{152, 57, 14}
\definecolor{HeaderOrangeE}{RGB}{111, 35, 0}

\definecolor{HeaderGoldA}{RGB}{249, 218, 121}
\definecolor{HeaderGoldB}{RGB}{214, 180, 72}
\definecolor{HeaderGoldC}{RGB}{187, 151, 37}
\definecolor{HeaderGoldD}{RGB}{152, 119, 14}
\definecolor{HeaderGoldE}{RGB}{111, 84, 0}

% Color triad based on HeaderBlue1
% http://paletton.com/
% Blues:
% 84, 125, 161 lightest blue
% 53, 98, 138 light blue
% 31,78,121 HeaderBlue1
% 16, 60, 99 darker blue
% 4, 40, 72 darkest blue
% Oranges:
% 249, 161, 121 lightest orange
% 214, 116, 72 light orange
% 187, 84, 37 main orange
% 152, 57, 14 darker orange
% 111, 35, 0 dark orange
% Golds:
% 249, 218, 121 lightest goldenrod
% 214, 180, 72 light goldenrod
% 187, 151, 37 goldenrod
% 152, 119, 14 dark goldenrod
% 111, 84, 0 darkest goldenrod

\definecolor{HyperlinkBlue1}{RGB}{64,172,209}
\definecolor{graphicOrange}{RGB}{255,153,51}
% Header1 is Arial(body)(Caps) 14pt
% Header2 is Arial(body) 12pt
% Body is arial 9pt
\usepackage{marginnote}
%\usepackage{geometry}
\usepackage[top=1.5in, bottom=1.5in, outer=2.5in, inner=1.25in, heightrounded, marginparwidth=1.5in, marginparsep=.25in]{geometry} % Sets page geometry
\usepackage{fancyhdr} % Adds control of headers and footers
%%%%  Testing
\usepackage{pifont}
\usepackage{anyfontsize}
\usepackage{multicol}
%%%%%%%%%%%%%


%\newcommand{\localtextbulletone}{\textcolor{HeaderOrangeC}{\raisebox{.45ex}{\rule{.6ex}{.6ex}}}}

% ding characters specified in symbols-a4 pg12
\newcommand{\localtextbulletone}{\textcolor{HeaderOrangeC}{\ding{227}}}
\renewcommand{\labelitemi}{\localtextbulletone}

\newcommand{\HRule}{\rule{\linewidth}{0.5mm}} % Command to make horizontal graphic lines
\newcommand{\marginText}{\fontfamily{cmr}\fontseries{b}\color{red}\fontsize{9}{12}\selectfont}
\newcommand{\helv}{\fontfamily{phv}\fontseries{b}\fontsize{9}{11}\selectfont}
% font has encoding, family,  series{b}(bold), shape, size{(size)}{(baselineskip)}
%       % https://www.latex-project.org/help/documentation/fntguide.pdf
%       % https://www.overleaf.com/learn/latex/Font_typefaces
%
\graphicspath{{img/}{GIS_ChampionSection/img/}{awardsChapter/GIS_ChampionSection/img/}{brandPart/awardsChapter/GIS_ChampionSection/img/}{img/}{pairedProgSection/img/}{methodChapter/pairedProgSection/img/}{methodPart/methodChapter/pairedProgSection/img/}{documentationSection/img/}{methodChapter/documentationSection/img/}{methodPart/methodChapter/documentationSection/img/}{docStorageOrgSection/img/}{methodChapter/docStorageOrgSection/img/}{methodPart/methodChapter/docStorageOrgSection/img/}{QGisSection/img/}{toolsChapter/QGisSection/img/}{servicePart/toolsChapter/QGisSection/img/}{ESRISection/img/}{toolChapter/ESRISection/img/}{servicePart/toolChapter/ESRISection/img/}{../../../../source/}{../../source/}{servicePart/applicationsChapter/treasurerSection/img/}{servicePart/toolsChapter/gisAdminSection/img/}{gisAdminSection/img/}{servicePart/toolsChapter/bsaSupportSection/img/}{servicePart/toolsChapter/coreDataSection/img/}{../img}}
%
%%%%%%%%%%%%%%%%%%%%%%%%%%%%%%%%%%%%%%%%%%%%%%%%%%%%%%%%%%%%%
%\usepackage{fancyhdr} % Adds control of headers and footers
%
%     Code Key: E : Even page, O : Odd page, L Left field,  C : Center field,
%               R : Right field, H : Header, F : Footer
%    \leftmark : Contains the Left argument of the last \markboth on the pages
%    \rightmark : Contains the Right Argument of the first \markboth of the page OR:
%                 the first /markright of the page
%
\renewcommand{\chaptermark}[1] % Redefinition of how chapter mark is displayed
{\markboth{\MakeUppercase{\thechapter.\ #1}}{}}
%
\renewcommand{\subsectionmark}[1] % Redefinition of how subsection mark is displayed
{\markright{\MakeUppercase{\thesubsection.\ #1}}}
%
\renewcommand{\headrulewidth}{0.5pt} % Line after header
\renewcommand{\footrulewidth}{0.5pt} % line before footer
%
\fancyhf{} % clear all header and footer fields
%
\fancyhead[LE,RO]{\helv \thepage} % Left Even and Right Odd headers
%                                 % \thepage macro displays the current page number
\fancyhead[LO]{\helv \leftmark} % Left Odd headers are set to /markboth
%                               % (chapters from \renewcommand{\chaptermark}[1]... )
\fancyhead[RE]{\helv \rightmark} % Right Even Headers
%                                % (subsection from \renewcommand{\subsectionmark}[1]...)




%\fancyheadoffset[RE,LO,RO,LE]{.5\textwidth}
%

% Alternate fancy header options
%\renewcommand{\sectionmark}[1] % Redefinition of how section mark is displayed
%{\markright{\MakeUppercase{\thesection.\ #1}}}
%
% Alternate fancy footer options
%\fancyfoot[C]{\textbf{\thepage}} % Page number on bottom center
%\fancyfoot[EC,OC]{\helv \thesection} % Section number bottom center

%%%%%%%%%%%%%%%%%%%%%%%%%%%%%%%%%%%%%%%%%%%%%%%%%%%%%%%%%%
% Test area

%\setlength\parindent{0pt} % eliminates indents
%\tolerance=1500 % Number affects overfull and Underful box warnings
%\usepackage{caption}

%\setlist[description]{leftmargin=\parindent,labelindent=\parindent}
\newcommand{\titleGeometry}{\geometry{top=1in,bottom=1in,inner=1.5in,outer=1.5in}} % Sets geometry for first part}

\usepackage[pdftex,breaklinks,colorlinks=true,linkcolor=black,citecolor=blue,urlcolor=red,linktocpage=false,pagebackref=true,filecolor=magenta]{hyperref}% http://www.tug.org/applications/hyperref/manual.html#x1-100003.6
\hypersetup{pdfauthor={Allegan County GIS Services},% Page 248 in Latex Beinners Guide
pdftitle={What We Do},
pdfsubject={Documentation of everything},
pdfkeywords={documentation,gis}}


\def\titlename{Forfeiture Data Collection\\ \medskip\large Mobile App with Collector for ArcGIS}

\title{\HRule % Horizontal Line added
\\[.4cm] % space
\protect\begin{figure}[H] % included image
\protect\begin{center}	% centered horizontally
\includegraphics[scale=.45]{GIS_Logo_better.jpg}
\protect\end{center}
\protect\end{figure}
\Huge \bfseries \titlename \\ % Title text
\HRule \\[.4cm] % Horizontal Line added
\author{\Large Allegan County GIS \\ \Large www.allegancounty.org/gis} % defines author
}  % inputs common title
\setcounter{tocdepth}{5}  % subparagraph and down
\begin{document}% document begins

\ifstandalone
\maketitle % creates title page
\clearpage
\tableofcontents % creates TOC
\clearpage
\fi

\subsection{Forfeiture Data Collection}

\subsubsection{Problem and Analysis}

\paragraph{Background}
Treasurer department has an annual responsibility to properly document the tax forfeiture process.  The LIS Department built an application in MS Access and MapInfo that consumed a daily export from BSA and was deployed to the field on a laptop.  A digital camera was used for site photos and later imported into the laptop.

\paragraph{Statement of Problem}
Current Tax Forfeiture workflow is built on MapInfo software which has been replaced by ESRI software.  The Forfeiture data collection application must be recreated in the ESRI framework.

\paragraph{Analysis}
Tax Forfeiture Application, referred to here as: \textbf{Forfeiture App} will facilitate:

\begin{itemize} %1

\item Mobile data collection on handheld device, referred to here as: \textbf{Mobile Interface}

\begin{itemize} %2

\item Mobile Interface will:

\begin{itemize} %3

\item Synchronize with data in the office (online)
\item Navigate to forfeiture sites (offline)
\item Collect data and photos of forfeiture sites (offline)
\item Synchronize the collected data with data in the office (online)
\end{itemize} %3

\end{itemize} %2

\item Daily form production and printing for each site visited with required data and images

\end{itemize} %1

\clearpage
\subsubsection{Design}
\paragraph{Overview}

The Forfeiture App documents the Tax Forfeiture process
\vspace{.25in}

The key data set, is referred to here as: \textbf{Forfeiture Parcels}
%An enterprise GIS deployment enables offline data collection by up to two users.
\begin{figure}[h!]
\centering
    \includegraphics[width=1\textwidth]{ProjectDesign}
\caption{Project Design}
\end{figure}

\clearpage

\paragraph{Forfeiture App Summary}
\vspace{.25in}

Three parts of the daily routine:
\begin{enumerate}
\item \Large Preprocessing \normalsize(in the office):

\begin{itemize}
\item Export current forfeiture list from BSA
\item Update Forfeiture Parcels with BSA export
\item Update Forfeiture Parcels with contaminated sites information
\item Synchronize Forfeiture Parcels to Mobile Interface
\end{itemize}

\item \Large Field data collection \normalsize with Mobile Interface:

\begin{itemize}
\item Aids in navigation
\item Provides a Checklist of data points for each site
\item Attaches photos for each site
\item Save results for synchronization in post-processing
\end{itemize}

\item \Large Post-processing \normalsize (in the office)

\begin{itemize}
\item Synchronize data and images collected in Mobile Interface to Forfeiture Parcels
\item Export form for each site
\item Print form for each site
\item Update BSA data

\end{itemize}
\end{enumerate}

\clearpage

\paragraph{Technologies Used in The Forfeiture App}
\subparagraph{BSA Data}Details of parcels in the forfeiture process are managed in BSA Delinquent Tax.net.  The Treasurer office does a BSA export of the parcels in need of a site visit in the preprocessing.

\subparagraph{ArcGIS Desktop}Tools are designed to preprocess and postprocess forfeiture parcel data for fieldwork.  The user will execute a preprocess script tool that prepares the data for field deployment.  After fieldwork, a post process script tool syncronizes data from the fieldwork with the live data on the Allegan County network.

\subparagraph{ArcGIS Collector}A free mobile application developed and tested on Android is deployed to the field for data collection.  The application is configured to work offline(without an internet or cellular connection) by syncronizing before and after fieldwork. The user collects the necessary information on each forfeiture parcel in the field disconnected, and then uploads the changes when reconnected.

\subparagraph{ArcGIS Enterprise Geodatabse}Live data from a publishing geodatabase (ACPub), running on SQL Server database server (acintsql01) provides access to Forfeiture Parcels

\subparagraph{ArcGIS Portal Webmaps and Apps} Forfeiture Parcels is served as a feature service (REST service)  named TaxReversionParcels.  A webmap on Portal, called the Forfeiture Field Map consumes the TaxReversionParcels exposing the data to editing.  The Forfeiture Field Map is configured to work in the ArcGIS Collector App.

\begin{figure}[h!]
\centering
    \includegraphics[width=.9\textwidth]{TechFlowChart}
\caption{Technology Design}
\end{figure}

\clearpage

\paragraph{Data Details}
\subparagraph*{Data Location}

\begin{wrapfigure}{r}{0.6\textwidth}
\centering
\includegraphics[width=.5\textwidth]{LiveDataLocation}
\caption{Live Data Location}
\vspace{.25in}

\HRule \\[.4cm] % Horizontal Line added
\vspace{.25in}

\includegraphics[width=.4\textwidth]{contaminationFeatureClass.png}
\caption{Contamination Feature Class}
\end{wrapfigure}
The data is located in ACPUB.  ACPUB is a geodatabase on ACINTSQL01.
\vspace{.5in}

Forfeiture Parcels Data
\vspace{4in}

Contamination Data
\clearpage
\vspace{.5in}

\paragraph{ForfeitureParcels Feature Class}Data Details:
\vspace{.5in}

\begin{table}[htbp]
\centering
\resizebox{\linewidth}{!}{%
\begin{tabular}{|l|l|c|r|}
\hline
\multicolumn{4}{|c|}{{\LARGE Attribute Details}} \\
\hline
Field Name&Field Alias&Entry Type&Note\\ \hline
PropertyNumber&Property Number&Prefilled&NA\\
Need2Print&Print Today&Dropdown&{\tiny Yes or No}\\
InspectionDate&Inspection Date&{\tiny Autofill or Dropdown}&NA\\
Inspector&Inspector&Dropdown&NA\\
Address&Address&Prefilled&NA\\
Status&Status&Dropdown&NA\\
StatusNotes&Status Notes&Open Entry&120Char\\
Roadfrontage&Road Frontage&Dropdown&{\tiny Yes or No}\\
AccessVia&Access Via&Open Entry&30Char\\
Agent&Agent&Open Entry&30Char\\
AgentContact&Agent Contact&Open Entry&30Char\\
PictureComments&Picture Comments&Open Entry&50Char\\
PropertyInUse&Property In Use&Dropdown&{\tiny Yes or No}\\
UseNotes&Use Notes&Open Entry&120Char\\
{\tiny PropertyMaintained}&{\tiny Property Maintained}&Dropdown&{\tiny Yes or No}\\
PropMaintNotes&{\tiny Property Maintained Notes}&Open Entry&120Char\\
{\tiny PropertyContaminated}&{\tiny Property Contaminated}&Prefilled&{\tiny Preprocessing}\\
{\tiny PropertyContaminatedNotes}&{\tiny PropertyContaminatedNotes}&Prefilled&{\tiny Preprocessing}\\
{\tiny AdjacentPropertyContaminated}&{\tiny Adjacent Property Contaminated}&Prefilled&{\tiny Preprocessing}\\
{\tiny AdjPropertyContaminatedNotes}&{\tiny Adj Property Contaminated Notes}&Prefilled&{\tiny Preprocessing}\\
PropertyForSale&Property For Sale&Dropdown&{\tiny Yes or No}\\
GlobalID&GlobalID&NA&NA\\
PostedDate&Posted Date&Dropdown&Date\\
Posted&Posted&Prefilled&NA\\
InList&In List&Prefilled&{\tiny Preprocessing}\\
PostedInList&Posted In List&Prefilled&{\tiny Preprocessing}\\
Acres&Acres&Prefilled&NA\\
Class&Class&Prefilled&NA\\
\hline
\end{tabular}
}
\caption{Dataset Details}
\end{table}
\clearpage
\paragraph{Webmap Details}The Forfeiture Field Map is made up of a basemap and a feature layer.
\begin{figure}[h!]
\centering
\includegraphics[width=.5\textwidth]{webMapDetails}
\caption{Web Map Details}
\end{figure}

\subparagraph{Feature Layer Details}TaxReversionParcels has been configured for offline use.
\begin{figure}[h!]
\centering
\includegraphics[width=.5\textwidth]{layerDetails}
\caption{Feature Layer Details}
\end{figure}
\clearpage
%
%images of publishing the service and explanation of settings required for offline use
%
\subparagraph{Basemap Details}
\begin{itemize}
  \setlength\itemsep{4em}
  \item A tiled basemap service is used
  \item The infoserv user credentials are used for sharing
  \item The url for the shared service is:
\end{itemize}

\begin{verbatim}
https://tiledbasemaps.arcgis.com/arcgis/rest/
   services/World_Street_Map/MapServer
\end{verbatim}
\begin{figure}[h!]
\centering
\includegraphics[width=1\textwidth]{BasemapSourceDescription}
\caption{Basemap Source Description}
\end{figure}
\clearpage
%
%
%
%
%
%
%Unplaced images
%\subparagraph{unplaced images}
%\begin{figure}[h!]
%\centering
%\includegraphics[width=.7\textwidth]{PortalContents}
%\caption{Portal Contents}
%\end{figure}
%\subparagraph{Feature layer configuration}
%\begin{figure}[h!]
%\centering
%\includegraphics[width=.7\textwidth]{featureLayerConfig}
%\caption{Feature Layer Configuration}
%\end{figure}
%\begin{figure}[h!]
%\centering
%\includegraphics[width=.5\textwidth]{FieldMapOnPC}
%\caption{Field Map on PC}
%\end{figure}
%
%\clearpage
\subsubsection{Hard Copy Record}
screenshots:
arcmap map
arcmap tools
portal screenshots
sql server mgt screen shots
phone screenshots

\paragraph{ArcGIS Server}

\clearpage
%
%
%
%
%
%
\subsubsection{Administrative Manual}
\vspace{.25in}

\paragraph{Annual Setup}
\vspace{.25in}

\noindent{\Large To Create the  new ForfeitureParcels dataset}
\vspace{.25in}

\noindent {\Large Use the Delete Feature Tools}
\vspace{.25in}

\noindent The tool will delete features in the feature class and attachments table}
\vspace{.3in}

\noindent In the tool: {\Large Select ACPub.DBO.ForfeitureParcels}
\vspace{.25in}

\noindent{\Large Press OK}
\vspace{.5in}

\begin{figure}[h!]
\centering
    \includegraphics[width=1\textwidth]{AnnualDeleteFeatures.png}
\caption{Delete Features}
\end{figure}
\clearpage
%
%
%
%
%
%
\subparagraph[Add Query Layer ]{\Large Add Query Layer\texorpdfstring{\\}{}}
\noindent {\Large In ArcMap:}
\vspace{.3in}

\noindent {\Large Open the New Query Layer Dialog}
\vspace{.3in}

{\LARGE Go to $\Rightarrow$ File $\Rightarrow$ Add Data $\Rightarrow$ Add Query Layer}
\vspace{.3in}

\noindent In the connection dropdown, select your connection
\begin{figure}[h!]
\centering
    \includegraphics[width=.95\textwidth]{NewQueryLayerDialog.png}
\caption{New Query Layer Dialog}
\end{figure}
\clearpage
%
%
%
%
%
%
\subparagraph{Create Query in ArcGIS to SQL Database}
\subparagraph[Details of the Query Layer]{\Large Details of the Query Layer}
\subparagraph*{Enter into the tool}
\begin{itemize}
  \item Choose connection
  \item Name the query
  \item Enter SQl query
  \item Press Next
\end{itemize}
\begin{figure}[h!]
\centering
    \includegraphics[width=.75\textwidth]{ForfeitureQueryLayerDetails.PNG}
\caption{Forfeiture Query Layer Details}
\end{figure}
\clearpage
%
%
%
%
%
%
\subparagraph[Unique Identifier]{\Large Select a Unique Identifier}
\subparagraph*{}
\begin{figure}[h!]
\centering
    \includegraphics[width=.95\textwidth]{ForfeitureQueryLayerUID.PNG}
\caption{Query Layer Unique ID}
\end{figure}
\clearpage
%
%
%
%
%
\paragraph[Setup Users in ArcGIS]{\Large Setup Users in ArcGIS\texorpdfstring{\\}{}}

Users that will run Pre and Post processing scripts must be created and given priviliges on ACPub Treasurer Feature Data Set.
\vspace{.5in}

\noindent For any new users of the geoprocessing tools, use the create Database User tool {\textbf or}
\vspace{.5in}

\noindent Go to $\Rightarrow$ Right click on ACpub $\Rightarrow$ Administration $\Rightarrow$ Add User
\begin{figure}[h!]
\centering
    \includegraphics[width=.9\textwidth]{addDbUser.png}
\caption{Add Db User}
\end{figure}
%
\clearpage
%
%
%
%
%
%
\paragraph[Add New User to Feature Dataset]{\Large Add New User to Feature Dataset\texorpdfstring{\\}{}}
\vspace{.5in}

In Catalog, $\Rightarrow$ right click on Treasurer Feature Data Set $\Rightarrow$ Manage $\Rightarrow$ Priviliges $\Rightarrow$ Add $\Rightarrow$ Type new user $\Rightarrow$ ok
\vspace{.5in}

\begin{figure}[h!]
\centering
    \includegraphics[width=.9\textwidth]{AddFdsUser.png}
\caption{Add Feature Dataset User}
\end{figure}
\clearpage
%
%
%
%
%
%
\paragraph[Extend Priviliges for New User]{\Large Extend Priviliges for New User\texorpdfstring{\\}{}}
\vspace{.5in}

In Catalog$\Rightarrow$right click on Treasurer FDS $\Rightarrow$ Manage$\Rightarrow$ Priviliges$\Rightarrow$ check boxes
\vspace{.5in}

\begin{figure}[h!]
\centering
    \includegraphics[width=.9\textwidth]{AddFdsUserPriviliges.png}
\caption{Extend Feature Dataset Priviliges}
\end{figure}

\clearpage
%
%
%
%
%
%
\paragraph[Setup Users in Portal for ArcGIS]{\Large Setup Users in Portal for ArcGIS\texorpdfstring{\\}{}}
\vspace{.5in}

\noindent Users that will use the Collector for ArcGIS must have profiles added to and managed in the Allegan County GIS Portal site.
\vspace{.5in}

In Portal go to My Organization
\begin{figure}[h!]
\centering
    \includegraphics[width=.9\textwidth]{PortalAddUser1.png}
\caption{Portal Add User 1}
\end{figure}

\clearpage
%
%
%
%
%
%
\paragraph[Add Members to Portal]{\Large Add Members to Portal\texorpdfstring{\\}{}}
\vspace{.5in}

Push add members $\Rightarrow$ built in member
\vspace{.5in}

\begin{figure}[h!]
\centering
    \includegraphics[width=.9\textwidth]{PortalAddUser2.png}
\caption{Portal Add User 2}
\end{figure}

\clearpage
%
%
%
%
%
%
\paragraph[Enter required info]{\Large Enter required info\texorpdfstring{\\}{}}

\vspace{.5in}

\begin{figure}[h!]
\centering
    \includegraphics[width=.9\textwidth]{PortalAddUser3.png}
\caption{Portal Add User 3}
\end{figure}

\clearpage
%
%
%
%
%
%
\paragraph[Manage Treasurer Group]{\Large Manage Treasurer Group\texorpdfstring{\\}{}}
\vspace{.5in}

In Portal $\Rightarrow$ Go to groups $\Rightarrow$ Invite new user to the group
\vspace{.5in}

\begin{figure}[h!]
\centering
    \includegraphics[width=.9\textwidth]{PortalAddUser4.png}
\caption{Portal Add User 4}
\end{figure}

\clearpage
%
%
%
%
%
%
\paragraph[Share Portal Content]{\Large Share Content To The Group\texorpdfstring{\\}{}}
\vspace{.5in}

\noindent Any content used by the group needs to be shared to the group
\vspace{.5in}

\begin{figure}[h!]
\centering
    \includegraphics[width=.9\textwidth]{PortalAddUser5.png}
\caption{Portal AddUser 5}
\end{figure}

\clearpage
%
%
%
%
%
%
\paragraph[Schema Change Procedure]{\Large Schema Change Procedure}
\clearpage
\paragraph[Form Edits Procedure]{\Large Form Edits Procedure}

\clearpage

\subsubsection[User Manual]{\Large User Manual}

\paragraph{Collection Device Setup}

\paragraph{Collector Application Setup Details}

\subparagraph{Install Collector for ArcGIS}
\begin{itemize}
\item Available from the Google Play Store
\end{itemize}
\begin{figure}[h!]
\centering
    \includegraphics[width=.65\textwidth]{DownloadtheApp.png}
\caption{Download the App}
\end{figure}

\clearpage
%
%
%
%
%
%
\subparagraph[Configure Collector]{\Large Configure Collector}

\subparagraph*{}
\begin{wrapfigure}{r}{0.5\textwidth}
\centering
\includegraphics[width=.3\textwidth]{CollectorConnection}
\caption{Collector Connection}
\vspace{.25in}

\HRule \\[.4cm] % Horizontal Line added
\vspace{.25in}

\includegraphics[width=.3\textwidth]{EnterCredentials.png}
\caption{Enter Credentials}
\end{wrapfigure}
for Organization Website, Type:
\vspace{.5in}

\begin{verbatim}
https://gis.allegancounty.org/
portal_webadaptor

\end{verbatim}
\large then:
\vspace{.5in}

\noindent Press \Large Continue
\vspace{2in}

\noindent Enter Credentials
\vspace{.5in}

\noindent \large then:
\vspace{.5in}

\noindent Press \Large SIGN IN
\clearpage
%
%
%
%
%
%
\subparagraph{Download the Forfeiture Field Map}
\noindent There are 3 different versions of the map
\subparagraph*{}
\begin{wrapfigure}{r}{0.5\textwidth}
\centering
\includegraphics[width=.4\textwidth]{CollectorChooseMap.png}
\caption{Collector Maps Menu}
\vspace{.25in}

\HRule \\[.4cm] % Horizontal Line added
\vspace{.25in}

\includegraphics[width=.3\textwidth]{ChooseWorkAreaLarge.png}
\caption{Choose Work Area (large)}
\end{wrapfigure}
\begin{itemize}
\item Forfeiture Field Map
\item Forfeiture Field Map For Photos
\item Forfeiture Field Map For Attributes
\end{itemize}
\vspace{.25in}

\noindent {\footnotesize The Download option indicates it is not on the device but is available for offline use}
\vspace{.25in}

\noindent {\large Choose a Map}
\vspace{.25in}

\noindent {\large Press Download}
\vspace{1in}

\noindent {\large Specify work area}
\vspace{.25in}

and press
\vspace{.25in}

map detail

\vspace{.25in}

\noindent {\footnotesize Note that a larger area takes longer to download but the basemap only needs to be downloaded once}

\clearpage
%
%
%
%
%
%
\subparagraph{Choose Map Detail}

\subparagraph*{}
\begin{wrapfigure}{r}{0.5\textwidth}
\centering
\includegraphics[width=.3\textwidth]{ChooseMapDetail.png}
\caption{Choose Map Detail}
\vspace{.25in}

\HRule \\[.4cm] % Horizontal Line added
\vspace{.25in}

\includegraphics[width=.3\textwidth]{MaponDevice.png}
\caption{Map on Device}
\end{wrapfigure}
Zoom into the level of detail desired.
\vspace{.5in}

\noindent Press Download
\vspace{3.5in}

\noindent This area is ready for field data collection.

\clearpage
%
%
%
%
%
%
\paragraph{Open Camera Application Setup Details}

\subparagraph{Install Open Camera}
\begin{itemize}
\item Available from the Google Play Store
\end{itemize}
\begin{figure}[h!]
\centering
    \includegraphics[width=.6\textwidth]{openCameraAppStore.png}
\caption{Open Camera from Google Play Store}
\end{figure}

\clearpage
\subparagraph{Configure Open Camera}

\subparagraph*{}
\begin{wrapfigure}{r}{0.5\textwidth}
\centering
\includegraphics[width=.3\textwidth]{findSettings.png}
\caption{Find Settings Menu}
\vspace{.25in}

\HRule \\[.4cm] % Horizontal Line added
\vspace{.25in}

\includegraphics[width=.3\textwidth]{settingsScreen.png}
\caption{Setting Screen}
\end{wrapfigure}
In the Open Camera Application:
\vspace{1in}

\noindent Press the gear shaped \Large Settings \normalsize button to go into the settings menu
\vspace{3in}

\noindent Press the \Large Photo Settings \normalsize button
\clearpage
\subparagraph*{Set Photo Resolution}
\subparagraph*{\\}
\begin{wrapfigure}{r}{0.5\textwidth}
\centering
\includegraphics[width=.3\textwidth]{photoSettings.png}
\caption{Photo Settings Menu}
\vspace{.25in}

\HRule \\[.4cm] % Horizontal Line added
\vspace{.25in}

\includegraphics[width=.3\textwidth]{cameraResolutionSetting.png}
\caption{Camera Resolution Setting}
\end{wrapfigure}
In \Large photo settings:
\vspace{1in}

\noindent Press the \Large Camera resolution \normalsize button
\vspace{3in}

\noindent Select \textbf{\LARGE 640 x 480}
\clearpage
\paragraph{Daily Preprocessing Routine}
\subparagraph{Execute Preprocessing Script}A tool in ArcGIS that:
\begin{itemize}
\item Exports current forfeiture list from BSA
\item Updates webmap layers with results from BSA export
\end{itemize}
\subparagraph*{\\}
\begin{wrapfigure}{r}{0.75\textwidth}
\centering
\includegraphics[width=.65\textwidth]{preprocess.png}
\caption{Processing Tools}
\end{wrapfigure}
In Catalog:
\vspace{1in}

\noindent Open the toolbox
\vspace{1in}

\noindent Open tool 1
\clearpage
\subparagraph{Synchronize the Forfeiture Field Map}
\subparagraph*{}
\begin{wrapfigure}{r}{0.5\textwidth}
\centering
\includegraphics[width=.3\textwidth]{MapDownloaded.png}
\caption{Map Downloaded}
\vspace{.25in}

\HRule \\[.4cm] % Horizontal Line added
\vspace{.25in}

\includegraphics[width=.3\textwidth]{MapSyncronized.png}
\caption{Map Synchronized}
\end{wrapfigure}
\Large Note the date and time:
\vspace{1.5in}

\noindent Press \LARGE Sync
\vspace{1.5in}

\Large Note the date and time:
\vspace{1.5in}

{\Large Map is synchronized}
\clearpage
\paragraph{Forfeiture Data Collection}
\subparagraph{Forfeiture Parcels Data Details}

Attributes are of four entry types:
\begin{itemize}
\item prefilled
\item autofill
\item dropdown
\item text box
\end{itemize}
For each site visited, select the desired parcel, push the edit button and collect attributes.
\clearpage
\subparagraph[Device 1 Field Operation]{Device 1 Field Operation\texorpdfstring{\\}{}}

\begin{wrapfigure}{r}{0.5\textwidth}
\centering
\includegraphics[width=.3\textwidth]{selectParcel.png}
\caption {Select Parcel}

\vspace{.25in}

\HRule \\[.4cm] % Horizontal Line added
\vspace{.25in}

\centering
\includegraphics[width=.3\textwidth]{parcelDetails.png}
\caption{Parcel Details}
\end{wrapfigure}

%\vspace{2in}

%{\Large Select a parcel}
Select a parcel

\vspace{4in}

{\normalsize Push the} {\Large edit} {\normalsize button}
\clearpage

\subparagraph*{Device 1 Field Operation Cont.}
\begin{wrapfigure}{r}{0.4\textwidth}
\centering
\includegraphics[width=.2\textwidth]{printToday.png}
\caption {Print Today Yes or No}
\vspace{.2in}

\HRule \\[.4cm] % Horizontal Line added
\vspace{.1in}

\includegraphics[width=.2\textwidth]{enterDate.png}
\caption{Enter Date}
\vspace{.1in}

\HRule \\[.4cm] % Horizontal Line added
\vspace{.2in}

\includegraphics[width=.2\textwidth]{selectInspector.png}
\caption{Select Inspector}
\end{wrapfigure}
Select Yes for Print Today
\vspace{2.5in}

\noindent Select Use Current or enter any date
\vspace{2in}

\noindent Select Inspector From Dropdown
\clearpage
\subparagraph*{Device 1 Field Operation Cont.\texorpdfstring{\\}{}}
\begin{wrapfigure}{r}{0.4\textwidth}
\centering
\includegraphics[width=.2\textwidth]{status.png}
\caption {Status}
\vspace{.1in}

\HRule \\[.4cm] % Horizontal Line added
\vspace{.1in}

\includegraphics[width=.2\textwidth]{statusNotes.png}
\caption{Status Notes}
\vspace{.1in}

\HRule \\[.4cm] % Horizontal Line added
\vspace{.1in}

\includegraphics[width=.2\textwidth]{roadFrontage.png}
\caption{Road Frontage}
\end{wrapfigure}
Select Occupied or Not Occupied
\vspace{2in}

\noindent Enter status notes up to 120 characters
\vspace{2in}

\noindent Select Yes or No for Road Frontage
\clearpage
\subparagraph*{Device 1 Field Operation Cont.\texorpdfstring{\\}{}}
\begin{wrapfigure}{r}{0.4\textwidth}
\centering
\includegraphics[width=.2\textwidth]{accessVia.png}
\caption {Access Via}
\vspace{.1in}

\HRule \\[.4cm] % Horizontal Line added
\vspace{.1in}

\includegraphics[width=.2\textwidth]{agent.png}
\caption{Agent}
\vspace{.1in}

\HRule \\[.4cm] % Horizontal Line added
\vspace{.1in}

\includegraphics[width=.2\textwidth]{agentContact.png}
\caption{Agent Contact}
\end{wrapfigure}
Enter road used for access
\vspace{2in}

\noindent Enter Agent Name
\vspace{3in}

\noindent Enter Agent Contact Info
\clearpage
\subparagraph*{Device 1 Field Operation Cont.\texorpdfstring{\\}{}}
\begin{wrapfigure}{r}{0.4\textwidth}
\centering
\includegraphics[width=.2\textwidth]{notes.png}
\caption {Use Notes}
\vspace{.1in}

\HRule \\[.4cm] % Horizontal Line added
\vspace{.1in}

\includegraphics[width=.2\textwidth]{notes.png}
\caption{Notes}
\vspace{.1in}

\HRule \\[.4cm] % Horizontal Line added
\vspace{.1in}

\includegraphics[width=.2\textwidth]{propertyForSale.png}
\caption{Property for Sale}
\end{wrapfigure}
Enter  Use Notes up to 120 characters
\vspace{2in}

\noindent Enter Notes up to 120 characters
\vspace{2.5in}

\noindent Enter property for sale yes or no
\clearpage
\subparagraph*{Device 1 Field Operation Cont.\texorpdfstring{\\}{}}
\begin{wrapfigure}{r}{0.4\textwidth}
\centering
\includegraphics[width=.2\textwidth]{propertyInUse.png}
\caption {Property in Use}
\vspace{.1in}

\HRule \\[.4cm] % Horizontal Line added
\vspace{.1in}

\includegraphics[width=.2\textwidth]{adjPropertyContNotes.png}
\caption{Placeholder}
\vspace{.1in}

\HRule \\[.4cm] % Horizontal Line added
\vspace{.1in}

\includegraphics[width=.2\textwidth]{notes.png}
\caption{Property Contaminated}
\end{wrapfigure}
Property in Use Yes or No
\vspace{2.5in}

\noindent Placeholder
\vspace{2.5in}

\noindent prefilled
\clearpage
\subparagraph*{Device 1 Field Operation Cont.\texorpdfstring{\\}{}}
\begin{wrapfigure}{r}{0.4\textwidth}
\centering
\includegraphics[width=.2\textwidth]{notes.png}
\caption {Notes up to 120 characters}
\vspace{.1in}

\HRule \\[.4cm] % Horizontal Line added
\vspace{.1in}

\includegraphics[width=.2\textwidth]{adjPropertyContNotes.png}
\caption{Adjacent Property Contaminated}
\vspace{.1in}

\HRule \\[.4cm] % Horizontal Line added
\vspace{.1in}

\includegraphics[width=.2\textwidth]{notes.png}
\caption{Property Contaminated}
\end{wrapfigure}
Enter notes up to 120 characters
\vspace{2in}

\noindent Adjacent Property Contaminated prefilled
\vspace{2in}

\noindent Property Contaminated notes prefilled
\clearpage
\subparagraph*{Device 1 Field Operation Cont.}
\begin{wrapfigure}{r}{0.4\textwidth}
\centering
\includegraphics[width=.2\textwidth]{propertyMaintained.png}
\caption {Property Maintained}
\vspace{.1in}

\HRule \\[.4cm] % Horizontal Line added
\vspace{.1in}

\includegraphics[width=.2\textwidth]{pictureComments.png}
\caption{Picture Comments}
\vspace{.1in}

\HRule \\[.4cm] % Horizontal Line added
\vspace{.1in}

\includegraphics[width=.2\textwidth]{notes.png}
\caption{Placeholder}
\end{wrapfigure}
Property Maintained Yes or No
\vspace{2.5in}

\noindent Picture Comments up to 120 characters
\vspace{2.5in}

\noindent Placeholder
\clearpage
\subparagraph[Device 2 Field Operation]{Device 2 Field Operation\texorpdfstring{\\}{}}

A photo or photos can be added from the Open Camera Application.

\begin{wrapfigure}{r}{0.4\textwidth}
\centering
\includegraphics[width=.2\textwidth]{selectParcel.png}
\caption {Select Parcel}
\vspace{.1in}

\HRule \\[.4cm] % Horizontal Line added

\vspace{.1in}

\includegraphics[width=.2\textwidth]{pushAttachmentButton.png}
\caption{Push Attachment Button}

\vspace{.1in}

\HRule \\[.4cm] % Horizontal Line added

\vspace{.1in}

\includegraphics[width=.2\textwidth]{addAttachmentFromGallery.png}
\caption{Add Attachment From Gallery}
\end{wrapfigure}

\vspace{.4in}

Select a parcel from the map
\vspace{2in}

\noindent Push Attachment Button
\vspace{2in}

\noindent Select Gallery

\clearpage
\subparagraph*{Device 2 Field Operation Cont.}

\begin{wrapfigure}{r}{0.4\textwidth}
\centering
\includegraphics[width=.2\textwidth]{openCameraFolderInGallery.png}
\caption {Open Camera Folder}
\vspace{.1in}

\HRule \\[.4cm] % Horizontal Line added
\vspace{.1in}

\includegraphics[width=.2\textwidth]{openCameraFolder.png}
\caption{In the Open Camera Folder}
\vspace{.1in}

\HRule \\[.4cm] % Horizontal Line added
\vspace{.1in}

\includegraphics[width=.2\textwidth]{imageInApp.png}
\caption{Image in the App}
\end{wrapfigure}
Navigate to the Open Camera Folder
\vspace{2in}

\noindent From within the Open Camera Folder, Select the appropriate image
\vspace{2in}

\noindent Press the check button to save the image to the parcel

\clearpage
\paragraph{Daily Postprocessing Routine}Back at the office
\subparagraph{Synchronize Webmap}In Collector for ArcGIS, push the sync button on the Forfeiture Field Map
\subparagraph{Execute Postprocessing Script}A tool in ArcGIS that:

\begin{itemize}
\item Reconciles geodatabase versions
\item Generates forms for each site visited

%Insert screenshot of Postprocesing script in Arc here

\end{itemize}

\clearpage
\subsubsection{Software}
\paragraph{ESRI Licensed Products}
\subparagraph{ArcDesktop}Users of this application need a license to ArcGIS Standard level.

\subparagraph{Enterprise ArcGIS Deployment}This app uses ArcGIS Server and ArcGIS Portal.

\subparagraph{Collector for ArcGIS}Developed and tested on Android(7.0).  Collector is available at the Google Play Store.

\paragraph{Other Software}

\subparagraph{Open Camera for Android}

\begin{figure}[h!]
\centering
    \includegraphics[width=.2\textwidth]{openCameraAppStore.png}
\caption{Open Camera from Google Play Store}
\end{figure}

\end{document}

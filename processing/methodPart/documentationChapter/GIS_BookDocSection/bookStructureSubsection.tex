%
%
%  Or Can compile alone with the commented texts toggled
%+++++++++++++++++++++++++++++++++++++++++++++++++++++++++++++++++++
%    To Do:
%
%
%
%-------------------------------------------------------------------
% OPTIONAL PREAMBLE FOR LOCAL COMPILE  %
%
% \def\titlename{How The Jalape\~no Works}
% \def\authorName{Allegan County GIS Services}
% \def\pdfTitle{How The Jalape\~no Works}
% \def\pdfSubject{GIS Tools} %
% \def\pdfKeywords{mobile,gis}
%
% %+++++++++++++++++++++++++++++++++++++++++++++++++++++++++++++++++++
%  Used in Subsections that are capable of being compiled alone
%
%
%-------------------------------------------------------------------
\documentclass[letterpaper, 12pt]{memoir}
 %


%%%%%%%%%%%%%%%%%%%%%%%%%%%%%%%%%%%%%%%%%%%%%%%%%%%%%%%%%%%%%
% Tried and True Preamble Section
\usepackage{import} % Required for importing other .tex docs.  (import uses everything bw Begin and End Doc)
\usepackage{float} % Required for specifying the exact location of a figure or table
\usepackage{wrapfig} % Provides environment to wrap figures with text
\usepackage{graphicx} % Required for including images
\usepackage{cite} % improves citation handling
%\usepackage[toc,title,page]{appendix}
\usepackage{pdfpages} % enables loading a pdf into the doc
\usepackage{makeidx} % Enables index features
\usepackage{glossaries} % must be after hyperref
\usepackage{blindtext} % enables ipsum loerem "dummy" text blocks
\usepackage{enumitem} % enables control over enumerate, itemize, and description
\usepackage[dvipsnames]{xcolor}%black, blue, brown, cyan, darkgray, gray, green, lightgray, lime, magenta, olive, orange, pink, purple, red, teal, violet, white, yellow.
%
\definecolor{HeaderBlueA1}{RGB}{10, 91, 163}
\definecolor{HeaderBlueA2}{RGB}{101, 98, 174}
\definecolor{HeaderBlueA3}{RGB}{84, 125, 161}
\definecolor{HeaderBlueB}{RGB}{53, 98, 138}
\definecolor{HeaderBlueC}{RGB}{31,78,121}  % HeaderBlue1
\definecolor{HeaderBlueD}{RGB}{16, 60, 99}
\definecolor{HeaderBlueE}{RGB}{4, 40, 72 }

\definecolor{HeaderOrangeA}{RGB}{249, 161, 121}
\definecolor{HeaderOrangeB}{RGB}{214, 116, 72}
\definecolor{HeaderOrangeC}{RGB}{187, 84, 37}
\definecolor{HeaderOrangeD}{RGB}{152, 57, 14}
\definecolor{HeaderOrangeE}{RGB}{111, 35, 0}

\definecolor{HeaderGoldA}{RGB}{249, 218, 121}
\definecolor{HeaderGoldB}{RGB}{214, 180, 72}
\definecolor{HeaderGoldC}{RGB}{187, 151, 37}
\definecolor{HeaderGoldD}{RGB}{152, 119, 14}
\definecolor{HeaderGoldE}{RGB}{111, 84, 0}

% Color triad based on HeaderBlue1
% http://paletton.com/
% Blues:
% 84, 125, 161 lightest blue
% 53, 98, 138 light blue
% 31,78,121 HeaderBlue1
% 16, 60, 99 darker blue
% 4, 40, 72 darkest blue
% Oranges:
% 249, 161, 121 lightest orange
% 214, 116, 72 light orange
% 187, 84, 37 main orange
% 152, 57, 14 darker orange
% 111, 35, 0 dark orange
% Golds:
% 249, 218, 121 lightest goldenrod
% 214, 180, 72 light goldenrod
% 187, 151, 37 goldenrod
% 152, 119, 14 dark goldenrod
% 111, 84, 0 darkest goldenrod

\definecolor{HyperlinkBlue1}{RGB}{64,172,209}
\definecolor{graphicOrange}{RGB}{255,153,51}
% Header1 is Arial(body)(Caps) 14pt
% Header2 is Arial(body) 12pt
% Body is arial 9pt
\usepackage{marginnote}
%\usepackage{geometry}
\usepackage[top=1.5in, bottom=1.5in, outer=2.5in, inner=1.25in, heightrounded, marginparwidth=1.5in, marginparsep=.25in]{geometry} % Sets page geometry
\usepackage{fancyhdr} % Adds control of headers and footers
%%%%  Testing
\usepackage{pifont}
\usepackage{anyfontsize}
\usepackage{multicol}
%%%%%%%%%%%%%


%\newcommand{\localtextbulletone}{\textcolor{HeaderOrangeC}{\raisebox{.45ex}{\rule{.6ex}{.6ex}}}}

% ding characters specified in symbols-a4 pg12
\newcommand{\localtextbulletone}{\textcolor{HeaderOrangeC}{\ding{227}}}
\renewcommand{\labelitemi}{\localtextbulletone}

\newcommand{\HRule}{\rule{\linewidth}{0.5mm}} % Command to make horizontal graphic lines
\newcommand{\marginText}{\fontfamily{cmr}\fontseries{b}\color{red}\fontsize{9}{12}\selectfont}
\newcommand{\helv}{\fontfamily{phv}\fontseries{b}\fontsize{9}{11}\selectfont}
% font has encoding, family,  series{b}(bold), shape, size{(size)}{(baselineskip)}
%       % https://www.latex-project.org/help/documentation/fntguide.pdf
%       % https://www.overleaf.com/learn/latex/Font_typefaces
%
\graphicspath{{img/}{GIS_ChampionSection/img/}{awardsChapter/GIS_ChampionSection/img/}{brandPart/awardsChapter/GIS_ChampionSection/img/}{img/}{pairedProgSection/img/}{methodChapter/pairedProgSection/img/}{methodPart/methodChapter/pairedProgSection/img/}{documentationSection/img/}{methodChapter/documentationSection/img/}{methodPart/methodChapter/documentationSection/img/}{docStorageOrgSection/img/}{methodChapter/docStorageOrgSection/img/}{methodPart/methodChapter/docStorageOrgSection/img/}{QGisSection/img/}{toolsChapter/QGisSection/img/}{servicePart/toolsChapter/QGisSection/img/}{ESRISection/img/}{toolChapter/ESRISection/img/}{servicePart/toolChapter/ESRISection/img/}{../../../../source/}{../../source/}{servicePart/applicationsChapter/treasurerSection/img/}{servicePart/toolsChapter/gisAdminSection/img/}{gisAdminSection/img/}{servicePart/toolsChapter/bsaSupportSection/img/}{servicePart/toolsChapter/coreDataSection/img/}{../img}}
%
%%%%%%%%%%%%%%%%%%%%%%%%%%%%%%%%%%%%%%%%%%%%%%%%%%%%%%%%%%%%%
%\usepackage{fancyhdr} % Adds control of headers and footers
%
%     Code Key: E : Even page, O : Odd page, L Left field,  C : Center field,
%               R : Right field, H : Header, F : Footer
%    \leftmark : Contains the Left argument of the last \markboth on the pages
%    \rightmark : Contains the Right Argument of the first \markboth of the page OR:
%                 the first /markright of the page
%
\renewcommand{\chaptermark}[1] % Redefinition of how chapter mark is displayed
{\markboth{\MakeUppercase{\thechapter.\ #1}}{}}
%
\renewcommand{\subsectionmark}[1] % Redefinition of how subsection mark is displayed
{\markright{\MakeUppercase{\thesubsection.\ #1}}}
%
\renewcommand{\headrulewidth}{0.5pt} % Line after header
\renewcommand{\footrulewidth}{0.5pt} % line before footer
%
\fancyhf{} % clear all header and footer fields
%
\fancyhead[LE,RO]{\helv \thepage} % Left Even and Right Odd headers
%                                 % \thepage macro displays the current page number
\fancyhead[LO]{\helv \leftmark} % Left Odd headers are set to /markboth
%                               % (chapters from \renewcommand{\chaptermark}[1]... )
\fancyhead[RE]{\helv \rightmark} % Right Even Headers
%                                % (subsection from \renewcommand{\subsectionmark}[1]...)




%\fancyheadoffset[RE,LO,RO,LE]{.5\textwidth}
%

% Alternate fancy header options
%\renewcommand{\sectionmark}[1] % Redefinition of how section mark is displayed
%{\markright{\MakeUppercase{\thesection.\ #1}}}
%
% Alternate fancy footer options
%\fancyfoot[C]{\textbf{\thepage}} % Page number on bottom center
%\fancyfoot[EC,OC]{\helv \thesection} % Section number bottom center

%%%%%%%%%%%%%%%%%%%%%%%%%%%%%%%%%%%%%%%%%%%%%%%%%%%%%%%%%%
% Test area

%\setlength\parindent{0pt} % eliminates indents
%\tolerance=1500 % Number affects overfull and Underful box warnings
%\usepackage{caption}

%\setlist[description]{leftmargin=\parindent,labelindent=\parindent}
\newcommand{\titleGeometry}{\geometry{top=1in,bottom=1in,inner=1.5in,outer=1.5in}} % Sets geometry for first part}

\usepackage[pdftex,breaklinks,colorlinks=true,linkcolor=black,citecolor=blue,urlcolor=red,linktocpage=false,pagebackref=true,filecolor=magenta]{hyperref}% http://www.tug.org/applications/hyperref/manual.html#x1-100003.6
\hypersetup{pdfauthor={Allegan County GIS Services},% Page 248 in Latex Beinners Guide
pdftitle={What We Do},
pdfsubject={Documentation of everything},
pdfkeywords={documentation,gis}}

 %
\graphicspath{{img/}{GIS_ChampionSection/img/}{awardsChapter/GIS_ChampionSection/img/}{brandPart/awardsChapter/GIS_ChampionSection/img/}{img/}{pairedProgSection/img/}{methodChapter/pairedProgSection/img/}{methodPart/methodChapter/pairedProgSection/img/}{documentationSection/img/}{methodChapter/documentationSection/img/}{methodPart/methodChapter/documentationSection/img/}{docStorageOrgSection/img/}{methodChapter/docStorageOrgSection/img/}{methodPart/methodChapter/docStorageOrgSection/img/}{QGisSection/img/}{toolsChapter/QGisSection/img/}{servicePart/toolsChapter/QGisSection/img/}{ESRISection/img/}{toolChapter/ESRISection/img/}{servicePart/toolChapter/ESRISection/img/}{../../../../source/}{../../source/}{servicePart/applicationsChapter/treasurerSection/img/}{servicePart/toolsChapter/gisAdminSection/img/}{gisAdminSection/img/}{servicePart/toolsChapter/bsaSupportSection/img/}{servicePart/toolsChapter/coreDataSection/img/}{../img}}

 %
%   This script does not require Memoir Class
%%%%%%%%%%%%%%%%%%%%%%%%%%%%%%%%%%%%%%%%%%%%%%%%%%%%%%%%%%%%
%+++++++++++++++++++++++++++++++++++++++++++++++++++++++++++++++++++
% Custom Color pallette
    % Blues
\definecolor{HeaderBlueA}{RGB}{85,125,161}
\definecolor{HeaderBlueB}{RGB}{53, 98, 138}
\definecolor{HeaderBlueC}{RGB}{31,78,121}  % HeaderBlue1
\definecolor{HeaderBlueD}{RGB}{16, 60, 99}
\definecolor{HeaderBlueE}{RGB}{4, 40, 72 }
    % Oranges
\definecolor{HeaderOrangeA}{RGB}{249, 161, 121}
\definecolor{HeaderOrangeB}{RGB}{214, 116, 72}
\definecolor{HeaderOrangeC}{RGB}{187, 84, 37}
\definecolor{HeaderOrangeD}{RGB}{152, 57, 14}
\definecolor{HeaderOrangeE}{RGB}{111, 35, 0}
    % Golds
\definecolor{HeaderGoldA}{RGB}{249, 218, 121}
\definecolor{HeaderGoldB}{RGB}{214, 180, 72}
\definecolor{HeaderGoldC}{RGB}{187, 151, 37}
\definecolor{HeaderGoldD}{RGB}{152, 119, 14}
\definecolor{HeaderGoldE}{RGB}{111, 84, 0}
    % Greens
    % option 1
% \definecolor{HeaderGreenA}{RGB}{14, 219, 76}
% \definecolor{HeaderGreenB}{RGB}{23, 179, 70}
% \definecolor{HeaderGreenC}{RGB}{31, 131, 61}
% \definecolor{HeaderGreenD}{RGB}{30, 92, 49}
% \definecolor{HeaderGreenE}{RGB}{21, 50, 30}
    % option 2 (triad of HeaderBlueC)
\definecolor{HeaderGreenA}{RGB}{82, 169, 136}
\definecolor{HeaderGreenB}{RGB}{49, 145, 109}
\definecolor{HeaderGreenC}{RGB}{25, 126, 88}
\definecolor{HeaderGreenD}{RGB}{10, 103, 68}
\definecolor{HeaderGreenE}{RGB}{0, 75, 47}
% Color triad based on HeaderBlue1
% http://paletton.com/

\definecolor{HyperlinkBlue1}{RGB}{64,172,209}
\definecolor{graphicOrange}{RGB}{255,153,51}
%-------------------------------------------------------------------

 %
%  Requires Memoir Class
%%%%%%%%%%%%%%%%%%%%%%%%%%%%%%%%%%%%%%%%%%%%%%%%%%%%%%%%%%%%%%%%%%%
    % Sets sectioning Layout Properties (SPACING AND TYPEFACE PROPERTIES)
      % FOR HEADINGS OF SECTIONS, SUBSECTIONS, SUBSUBSECTIONS, PARAGRAPH AND SUBPARAGRAPH
      %
\setsecnumdepth{subsection} % turns on subsec numbering
      %
%+++++++++++++++++++++++++++++++++++++++++++++++++++++++++++++++++++
%  SET HEAD STYLES
\setsecheadstyle{\scshape\color{HeaderOrangeA}} % default is \Large\bfseries
    %\setsecheadstyle{\scshape\color{HeaderOrangeB}}
    %
    % Subsection
\setsubsecheadstyle{\large\scshape\color{HeaderOrangeB}} % default is \large\bfseries
    %\setsubsecheadstyle{\huge\centering\color{HeaderOrangeE}}%
    % Subsubsection
\setsubsubsecheadstyle{\Huge\scshape\centering\color{HeaderBlueE}} % default is \bfseries
    %\setsubsubsecheadstyle{\LARGE\scshape\centering\color{HeaderBlueC}}%   1
    % Paragraph
\setparaheadstyle{\bfseries\Large\color{HeaderBlueC}}%
     %\setparaheadstyle{\Large\color{HeaderBlueA}}%2
     % Subparagraph
     %\setsubparaheadstyle{\large\color{HeaderBlueB}}%
\setsubparaheadstyle{\bfseries\large\color{HeaderOrangeB}}%
%-------------------------------------------------------------------
  %
%+++++++++++++++++++++++++++++++++++++++++++++++++++++++++++++++++++
  %  SET BEFORE SKIPS (BW the title and the text)
     % \setbeforeSskip{<skip>} pg .95
     % The absolute value of the <skip> length argument is the space to leave above the heading.
     % If the actual value is negative then the first line after the heading will not be indented.
\setbeforesecskip{-35pt plus -10pt minus -2pt}
\setbeforesubsecskip{-32.5pt plus -10pt minus -2pt}
\setbeforesubsubsecskip{-32.5pt plus -10pt minus -2pt}
\setbeforeparaskip{-32.5pt plus -10pt minus -2pt}
\setbeforesubparaskip{-5pt plus -3pt minus -1pt}
%-------------------------------------------------------------------
  %
%+++++++++++++++++++++++++++++++++++++++++++++++++++++++++++++++++++
  %  SET BEFORE INDENTS
    % \setSindent{<length>}
    % The value of the length argument is the indentation of the heading (number and
    % title) from the lefthand margin. This is normally 0pt.
\setsubparaindent{0pt}
%-------------------------------------------------------------------
  %
%+++++++++++++++++++++++++++++++++++++++++++++++++++++++++++++++++++
  %  SET AFTER SKIPS (After the text)
    % \setafterSskip{<skip>} pg.96
    % If the value of the <skip> length argument is positive it is the space to leave between the
    % display heading and the following text. If it is negative, then the heading will be runin
    % and the value is the horizontal space between the end of the heading and the following text.
\setaftersecskip{.15in}
\setaftersubsecskip{.15in}
\setaftersubsubsecskip{.1in}
\setafterparaskip{.05in}
\setaftersubparaskip{.05in}
%-------------------------------------------------------------------
  %
%+++++++++++++++++++++++++++++++++++++++++++++++++++++++++++++++++++
  %  REDEFINE PART NAME AND NUMBER FONT
\renewcommand{\partnamefont}{\color{HeaderOrangeE}\Huge\normalfont}
\renewcommand{\partnumfont}{\color{HeaderOrangeE}\Huge\normalfont}
%-------------------------------------------------------------------
%
\setlength\columnsep{.3in} % Space BW columns in 2 col mode

 %
%   This script requires Memoir Class
%%%%%%%%%%%%%%%%%%%%%%%%%%%%%%%%%%%%%%%%%%%%%%%%%%%%%%%%%%%%
    %  Page Styles
    %  In LATEX the page style refers to the part of the page building mechanism that
    %   attaches the headers and footers to the actual page.
%%%%%%%%%%%%%%%%%%%%%%%%%%%%%%%%%%%%%%%%%%%%%%%%%%%%%%%%%%%%
%  This block defines jalapenoPageStyleA page style
\makepagestyle{jalapenoPageStyleA}
    % headers
\makeheadrule {jalapenoPageStyleA}{\textwidth}{\normalrulethickness}
\makeheadfootruleprefix{jalapenoPageStyleA}{\color{HeaderOrangeE}}{\color{HeaderOrangeE}}
\makeevenhead {jalapenoPageStyleA}{\helv \thepage}{} {\helv \rightmark}
\makeoddhead {jalapenoPageStyleA}{\helv \leftmark}{}{\helv \thepage}
    % footers
\makefootrule {jalapenoPageStyleA}{\textwidth}{\normalrulethickness}{\footruleskip}
\makeevenfoot {jalapenoPageStyleA}{}{} {}
\makeoddfoot {jalapenoPageStyleA}{}{} {}
    %
%%%%%%%%%%
    %
%  This block defines marks for jalapenoPageStyleA page style
    %  See page 11 of Page Styles in Memoir
    %
\makeatletter % because of \@chapapp
%
\makepsmarks {jalapenoPageStyleA}{
%\nouppercaseheads
\createmark {chapter} {both} {shownumber}{\@chapapp\ }{. \ } % set leftmark to chapter
%\createmark {section} {right}{shownumber}{} {. \ }
\createmark {subsection} {right}{shownumber}{} {. \ } % set rightmark to subsection
%\createmark {subsubsection}{right}{shownumber}{} {. \ } % set rightmark to subsubsection
\createplainmark {toc} {both} {\contentsname}
\createplainmark {lof} {both} {\listfigurename}
\createplainmark {lot} {both} {\listtablename}
\createplainmark {bib} {both} {\bibname}
\createplainmark {index} {both} {\indexname}
\createplainmark {glossary} {both} {\glossaryname}

}
\makeatother
%  End of jalapenoPageStyleA pagestyle
%%%%%%%%%%%%%%%%%%%%%%%%%%%%%%%%%%%%%%%%%%%%%%%%%%%%%%%%%%%%

%%%%%%%%%%%%%%%%%%%%%%%%%%%%%%%%%%%%%%%%%%%%%%%%%%%%%%%%%%%%
%  This block defines emptyPageStyle page style
\makepagestyle{emptyPageStyle}
    % headers
\makeevenhead {emptyPageStyle}{}{}{}
\makeoddhead {emptyPageStyle}{}{}{}
    % footers
\makefootrule {emptyPageStyle}{}{}{}
\makeevenfoot {emptyPageStyle}{}{}{}
\makeoddfoot {emptyPageStyle}{}{}{}
    %
%%%%%%%%%%

 %
%   This script requires Memoir Class
%%%%%%%%%%%%%%%%%%%%%%%%%%%%%%%%%%%%%%%%%%%%%%%%%%%%%%%%%%%%%%%%%%%%%%%%%%%%
  % beginning od customchpstyle
    %
\makeatletter % handler for @
    %
\makechapterstyle{jalapenoChapterStyle}{%
    %
    \newlength{\chapColorbarheight}
    \setlength{\chapColorbarheight}{.06in}		% Setting the height of the bar
    %
    \newlength{\chapnumboxlength}
    \setlength{\chapnumboxlength}{.75in}	% Setting the length of the box containing chapter number
    %
    \newlength{\leftbarlength}
    \setlength{\leftbarlength}{.5in}  		% Setting length of the bar left to the chapter number
    \setlength{\afterchapskip}{.5in}
    \renewcommand*{\chapterheadstart}{\vspace*{.25in}} % vspace from top of page
    \renewcommand*{\afterchapternum}{\vspace*{-.9in}} % neg moves chapter title above chapter number
    \renewcommand*{\chapnumfont}{\slshape} % Number Font
    \renewcommand*{\chaptitlefont}{\flushright\color{HeaderOrangeE}\slshape\huge}
    \renewcommand*{\printchaptername}{} % sets the text in brackets before the number
    %\makeatletter
    \newcommand*{\thickrulefill}{\leavevmode \leaders \hrule height \chapColorbarheight \hfill \kern \z@} % rightbar length setter
    %\makeatother
    \renewcommand*{\printchapternum}{%
        %\makebox[0pt][l]{%
        \color{HeaderOrangeE}
        \rule{\leftbarlength}{\chapColorbarheight} % Left bar
        \quad % hspace
        \resizebox{!}{\chapnumboxlength}
        {\chapnumfont \thechapter} % Chapter name and number
        \quad % hspace
        \color{HeaderOrangeE}\thickrulefill % implements thickrulefill
     } %
     \makeatother % closes handler for @
} % End of customchpstyle
 %\makeatother % closes handler for @
%%%%%%%%%%%%%%%%%%%%%%%%%%%%%%%%%%%%%%%%%%%%%%%%%%%%%%%%%%%%%%%%%%%%%%%%%%%%

 %
%\newglossaryentry{ex}{name={sample},description={an example}}


\newglossaryentry{projection}{name={map projection},description={Representing a sphere on a flat surface}}
 %
\newlength\drop
\makeatletter
\newcommand*\titleM{\begingroup% Misericords, T&H p 153
\setlength\drop{0.08\textheight}
\centering
\vspace*{\drop}
{\protect\HRule}
\begin{figure}[H] % included image
\begin{center}	% centered horizontally
\includegraphics[scale=.6]{GIS_Logo_better.jpg}
\end{center}
\end{figure}
\vspace{-.1in}
{\Huge\bfseries\titlename}\\[2\baselineskip]
{\protect\HRule}\\[\baselineskip]
{\small\scshape www.allegancounty.org/gis}\\[2\baselineskip]
{\small\scshape \@date}\par
\endgroup}
\makeatother
  % inputs common title
  % SET PDF METADATA  %
\hypersetup{pdfauthor={\authorName},
pdftitle={\pdfTitle}, %  Sets PDF properties
pdfsubject={\pdfSubject},
pdfkeywords={\pdfKeywords}}
%-------------------------------------------------------------------
  % TOC DEPTH  %
\setcounter{tocdepth}{4}  % Sets Table of Contents level to show subsections, sections, chapters, and parts(DEFAULT)
 %
%+++++++++++++++++++++++++++++++++++++++++++++++++++++++++++++++++++
%    SET TEXT BLOCK AND MARGINS FOR COVER PAGE  %
\setmarginnotes{.1in}{.4in}{.1in}
\setlrmarginsandblock{*}{0.18\paperwidth}{1} % Left right ratio
\setulmarginsandblock{1in}{1in}{*}
\checkandfixthelayout
\makeatletter
\ch@ngetext
\makeatother
  %+++++++++++++++++++++++++++++++++++++++++++++++++++++++++++++++++++
	%  Front Section
  %-------------------------------------------------------------------
\begin{document}% document begins
   %
\frontmatter % turns off chapter numbering and uses roman numerals for page numbers
   %
\pagestyle{empty} % Clear headers and footers for TOC
   %
\begin{titlingpage}
   %
\titleM  % Inputs titleM
   %
\end{titlingpage}
   %
%+++++++++++++++++++++++++++++++++++++++++++++++++++++++++++++++++++
%    SET TEXT BLOCK AND MARGINS FOR MAIN DOCUMENT PAGES   %
\setmarginnotes{.1in}{.4in}{.1in}
\setlrmarginsandblock{*}{0.18\paperwidth}{.75} % Left right ratio
\setulmarginsandblock{1in}{1in}{*}
\checkandfixthelayout
\makeatletter
\ch@ngetext
\makeatother
%-------------------------------------------------------------------
\tableofcontents % creates TOC
  %
%+++++++++++++++++++++++++++++++++++++++++++++++++++++++++++++++++++
%		Main Section
%-------------------------------------------------------------------
\mainmatter % turns on chapter numbering, resets page numbering and uses arabic numerals for page numbers
  %
\chapterstyle{jalapenoChapterStyle} % custom from chapterStyles.tex
  %
\pagestyle{jalapenoPageStyleA} % custom from pageStyles.tex

%
%\def\titlename{How The Jalape\~no Works}
%
%-------------------------------------------------------------------
    %
    %
\begin{document}% document begins
 %
 %-------------------------------------------------
\subsection{How Jalape\~no Works}
  %
\subsubsection{Problem and Analysis}
  %
\begin{adjmulticols}{2}{\innerMar}{\outerMar}
  %
\paragraph{Background}
  %
\noindent GIS Services has complicated and evolving workflows and uses everchanging technologies
  %
\paragraph{Statement of Problem}
  %
\noindent GIS documentation has traditionally been done in different formats and stored in many different files and folders in the county network.  This has resulted in problems with:
  %
\begin{itemize}
\item version control
\item finding the documentation
\item disseminating the documentation
\end{itemize}
  %
\paragraph{Analysis}
  %
\noindent The Jalape\~no folder along with some opensource software provides a robust documentaion tool for GIS documentation.
  %
\end{adjmulticols}
  %
\clearpage

\paragraph{Default sizes in Jalape\~no}

\begin{table}[htbp]

\centering

\resizebox{.7\linewidth}{!}{%
\begin{tabular}{|l|l|}
\hline
Element&Default Size\\ \hline
Paragraph Heading&Large\\ \hline
Paragraph text&normalsize\\ \hline
Subparagraph Heading&large\\ \hline
Subparagraph Text&normalsize\\ \hline
\end{tabular}
}
\caption{Default Sizes}
\end{table}

Examples:
\vspace{-.3in}

\paragraph*{\large Schema Change Procedure large size}

{\large large size type}
%\vspace{-.3in}

\paragraph*{Schema Change Procedure Default size}

default size type
%\vspace{-.3in}

\paragraph*{\Large Schema Change Procedure Large size}

{\Large Large size type}
space neg pouint 3in here
\vspace{-.3in}

\paragraph*{\LARGE Schema Change Procedure Large size}

{\LARGE LARGE size type}
\vspace{.1in}

\subparagraph{Schema Change Procedure Default size}

default size type

\subparagraph*{\large Schema Change Procedure large size}

{\large large size type}

\subparagraph*{\Large Schema Change Procedure Large size}

{\Large Large size type}

\subparagraph*{\LARGE Schema Change Procedure LARGE size}

{\LARGE LARGE size type}

\clearpage

  %
\subsubsection{Colors}
  %
\paragraph[Blues]{Blues\texorpdfstring{\\}{}}
\textcolor{HeaderBlueA}{HeaderBlueA}
\noindent\textcolor{HeaderBlueA}{\rule{.5\textwidth}{.5mm}}\\
\textcolor{HeaderBlueB}{HeaderBlueB}
\noindent\textcolor{HeaderBlueB}{\rule{.5\textwidth}{.5mm}}\\
\textcolor{HeaderBlueC}{HeaderBlueC}
\noindent\textcolor{HeaderBlueC}{\rule{.5\textwidth}{.5mm}}\\
\textcolor{HeaderBlueD}{HeaderBlueD}
\noindent\textcolor{HeaderBlueD}{\rule{.5\textwidth}{.5mm}}\\
\textcolor{HeaderBlueE}{HeaderBlueE}
\noindent\textcolor{HeaderBlueE}{\rule{.5\textwidth}{.5mm}}\\
  %
\paragraph[Golds]{Golds\texorpdfstring{\\}{}}
\textcolor{HeaderGoldA}{HeaderGoldA}
\noindent\textcolor{HeaderGoldA}{\rule{.5\textwidth}{.5mm}}\\
\textcolor{HeaderGoldB}{HeaderGoldB}
\noindent\textcolor{HeaderGoldB}{\rule{.5\textwidth}{.5mm}}\\
\textcolor{HeaderGoldC}{HeaderGoldC}
\noindent\textcolor{HeaderGoldC}{\rule{.5\textwidth}{.5mm}}\\
\textcolor{HeaderGoldD}{HeaderGoldD}
\noindent\textcolor{HeaderGoldD}{\rule{.5\textwidth}{.5mm}}\\
\textcolor{HeaderGoldE}{HeaderGoldE}
\noindent\textcolor{HeaderGoldE}{\rule{.5\textwidth}{.5mm}}\\
  %
\paragraph[Oranges]{Oranges\texorpdfstring{\\}{}}
\textcolor{HeaderOrangeA}{HeaderOrangeA}
\noindent\textcolor{HeaderOrangeA}{\rule{.5\textwidth}{.5mm}}\\
\textcolor{HeaderOrangeB}{HeaderOrangeB}
\noindent\textcolor{HeaderOrangeB}{\rule{.5\textwidth}{.5mm}}\\
\textcolor{HeaderOrangeC}{HeaderOrangeC}
\noindent\textcolor{HeaderOrangeC}{\rule{.5\textwidth}{.5mm}}\\
\textcolor{HeaderOrangeD}{HeaderOrangeD}
\noindent\textcolor{HeaderOrangeD}{\rule{.5\textwidth}{.5mm}}\\
\textcolor{HeaderOrangeE}{HeaderOrangeE}
\noindent\textcolor{HeaderOrangeE}{\rule{.5\textwidth}{.5mm}}\\
  %
\paragraph[Geens]{Greens\texorpdfstring{\\}{}}
\textcolor{HeaderGreenA}{HeaderGreenA}
\noindent\textcolor{HeaderGreenA}{\rule{.5\textwidth}{.5mm}}\\
\textcolor{HeaderGreenB}{HeaderGreenB}
\noindent\textcolor{HeaderGreenB}{\rule{.5\textwidth}{.5mm}}\\
\textcolor{HeaderGreenC}{HeaderGreenC}
\noindent\textcolor{HeaderGreenC}{\rule{.5\textwidth}{.5mm}}\\
\textcolor{HeaderGreenD}{HeaderGreenD}
\noindent\textcolor{HeaderGreenD}{\rule{.5\textwidth}{.5mm}}\\
\textcolor{HeaderGreenE}{HeaderGreenE}
\noindent\textcolor{HeaderGreenE}{\rule{.5\textwidth}{.5mm}}\\
  %
\paragraph[Others]{Others\texorpdfstring{\\}{}}
\textcolor{HyperlinkBlue1}{HyperlinkBlue1}
\noindent\textcolor{HyperlinkBlue1}{\rule{.5\textwidth}{.5mm}}\\
\textcolor{graphicOrange}{graphicOrange}
\noindent\textcolor{graphicOrange}{\rule{.5\textwidth}{.5mm}}\\
  %
\clearpage
  %
\subsubsection{Project Notes:}
  %
\begin{itemize}
%\item jalapeno folder is a git package
\item jalapeno folder is a \gls{git} package
  %
\href{https://github.com/nbesteman/jalapeno}{https://github.com/nbesteman/jalapeno}
  %
\item Project is coded with relative paths and jalapeno can be located anywhere.
  %
\end{itemize}
  %
\begin{adjmulticols}{1}{\innerMar}{\outerMar}
  %
\paragraph{Project File Structure:}
  %
\subparagraph*{...\textbackslash jalapeno\textbackslash..}
  %
\begin{tabular}{p{6cm}| p{9cm} }
\footnotesize folder & {\footnotesize description} \\ \hline
.git & versioning repository for Jalape\~no\\
documentation & resources used in Jalape\~no\\
processing & .tex douments and build folders\\
source & common image files\\
temp & untracked folder for temp storage\\
\end{tabular}
  %
\subparagraph*{...\textbackslash jalapeno\textbackslash documentation\textbackslash..}
  %
\begin{tabular}{p{6cm} | p{9cm} }
\footnotesize folder or file & {\footnotesize description} \\ \hline
classDocs & \LaTeX{} class documentation\\
DevNotes & Notes and Mind Maps for Jalape\~no\\
latexamples & \LaTeX{} example code\\
moduleTemplates & .tex templates\\
packageDocs & \LaTeX{} package documentation\\
readingRoom & Resources linked in Jalape\~no\\
unsorted & Unsorted documentation\\
gitnotes.txt & git commands notes\\
\end{tabular}
  %
\subparagraph*{}
  %
\subparagraph*{...\textbackslash jalapeno\textbackslash processing\textbackslash..}
  %
\begin{tabular}{p{6cm}| p{9cm} }
{\footnotesize folder or file} & {\footnotesize description }\\ \hline
archive & Processing backup folder\\
...Part & Folders of book \textit{part}s\\
build & \LaTeX{}folder for .pdf output and temp files \\
build\textbackslash referenceEntries.bib & Entries that appear in references\\
preamble.tex & preamble code for all documents\\
titlePages & Assortment of .tex title pages\\
compileFull.sh & {\scriptsize pdflatex, bibtex, makeglossaries, makeindex, pdflatex, pdflatex}\\
compileMainX2.sh & pdflatex, pdflatex\\
GISDocumentation.tex & Master document code\\
glossaryEntries.tex & Entries that appear in glossary\\
indexEntries.tex & Entries that appear in the index\\
  %
\end{tabular}
  %
\subparagraph*{...\textbackslash jalapeno\textbackslash processing\textbackslash preamble..}
  %
\begin{tabular}{p{6cm}| p{9cm} }
{\footnotesize folder or file} & {\footnotesize description }\\ \hline
chapterStyles.tex & {\scriptsize Sets chapter title page attributes with Memoir Class}\\
colorDefs.tex & Defines custom colors\\
graphicsPath.tex & Defines graphics variable\\
pageLayoutCommands.tex & {\scriptsize Sets spacing and typeface for headings of Sections down to Subparagraphs in mainmatter}\\
pageLayoutCommandsAlt.tex & {\scriptsize Sets spacing and typeface for headings of Sections down to Subparagraphs in backmatter}\\
pageStyles.tex & Sets header and footer properties\\
preamble.tex & {\scriptsize Preamble used to compile main document}\\
subSectionPreamble.tex & {\scriptsize Preamble used to compile any subsection document}\\
  %
\end{tabular}
  %
\end{adjmulticols}
  %
\clearpage

\subsubsection[Using The Glossary]{Using The Glossary}
  %
  %
\paragraph{Glossary Requirements}
Glossary commands require a Perl interpreter.  Activeperl is a free Perl interpreter and can be downloaded from:\\ \href{https://www.activestate.com/activeperl/downloads}{https://www.activestate.com/activeperl/downloads}
{\tiny (A typical installation adds Perl to your path)}.  Compiling the glossary requires running the makeglossaries command either in a \LaTeX{} IDE or in command line as described here.  PDFLatex must be run first to create a .aux file that is used by makeglossaries to create an .gls file.  After the .gls file is created, PDFLatex must be run again to insert the glossary at the \textbackslash printglossaries location.
  %
\paragraph{Creating a new glossary entry}
To \textbf{create a new glossary entry:} Add an entry to glossaryEntries.tex.  Save it there and then use the makeglossaries command to recompile the .gls file.
  %
\paragraph{Rebuilding the glossary}
\textbf{To Recompile the .gls}.  In the (main document)build folder:
  %
\begin{itemize}
\item Launch command prompt
\item enter command: \textbf{{\large makeglossaries GISDocumentation*}}
\end{itemize}
  %
\noindent{\smallbtn Note that this command reads the .aux file and creates the .gls file.  The .aux file is created by compiling with PDFLatex.  If there is no .aux file the command will fail}
  %
\paragraph{Using glossary terms in a subdocument:}
  %
In the subdocument you must add code to input the glossaryEntries file.\\
ie. After the line:
  %
\begin{verbatim}


%%%%%%%%%%%%%%%%%%%%%%%%%%%%%%%%%%%%%%%%%%%%%%%%%%%%%%%%%%%%%
% Tried and True Preamble Section
\usepackage{import} % Required for importing other .tex docs.  (import uses everything bw Begin and End Doc)
\usepackage{float} % Required for specifying the exact location of a figure or table
\usepackage{wrapfig} % Provides environment to wrap figures with text
\usepackage{graphicx} % Required for including images
\usepackage{cite} % improves citation handling
%\usepackage[toc,title,page]{appendix}
\usepackage{pdfpages} % enables loading a pdf into the doc
\usepackage{makeidx} % Enables index features
\usepackage{glossaries} % must be after hyperref
\usepackage{blindtext} % enables ipsum loerem "dummy" text blocks
\usepackage{enumitem} % enables control over enumerate, itemize, and description
\usepackage[dvipsnames]{xcolor}%black, blue, brown, cyan, darkgray, gray, green, lightgray, lime, magenta, olive, orange, pink, purple, red, teal, violet, white, yellow.
%
\definecolor{HeaderBlueA1}{RGB}{10, 91, 163}
\definecolor{HeaderBlueA2}{RGB}{101, 98, 174}
\definecolor{HeaderBlueA3}{RGB}{84, 125, 161}
\definecolor{HeaderBlueB}{RGB}{53, 98, 138}
\definecolor{HeaderBlueC}{RGB}{31,78,121}  % HeaderBlue1
\definecolor{HeaderBlueD}{RGB}{16, 60, 99}
\definecolor{HeaderBlueE}{RGB}{4, 40, 72 }

\definecolor{HeaderOrangeA}{RGB}{249, 161, 121}
\definecolor{HeaderOrangeB}{RGB}{214, 116, 72}
\definecolor{HeaderOrangeC}{RGB}{187, 84, 37}
\definecolor{HeaderOrangeD}{RGB}{152, 57, 14}
\definecolor{HeaderOrangeE}{RGB}{111, 35, 0}

\definecolor{HeaderGoldA}{RGB}{249, 218, 121}
\definecolor{HeaderGoldB}{RGB}{214, 180, 72}
\definecolor{HeaderGoldC}{RGB}{187, 151, 37}
\definecolor{HeaderGoldD}{RGB}{152, 119, 14}
\definecolor{HeaderGoldE}{RGB}{111, 84, 0}

% Color triad based on HeaderBlue1
% http://paletton.com/
% Blues:
% 84, 125, 161 lightest blue
% 53, 98, 138 light blue
% 31,78,121 HeaderBlue1
% 16, 60, 99 darker blue
% 4, 40, 72 darkest blue
% Oranges:
% 249, 161, 121 lightest orange
% 214, 116, 72 light orange
% 187, 84, 37 main orange
% 152, 57, 14 darker orange
% 111, 35, 0 dark orange
% Golds:
% 249, 218, 121 lightest goldenrod
% 214, 180, 72 light goldenrod
% 187, 151, 37 goldenrod
% 152, 119, 14 dark goldenrod
% 111, 84, 0 darkest goldenrod

\definecolor{HyperlinkBlue1}{RGB}{64,172,209}
\definecolor{graphicOrange}{RGB}{255,153,51}
% Header1 is Arial(body)(Caps) 14pt
% Header2 is Arial(body) 12pt
% Body is arial 9pt
\usepackage{marginnote}
%\usepackage{geometry}
\usepackage[top=1.5in, bottom=1.5in, outer=2.5in, inner=1.25in, heightrounded, marginparwidth=1.5in, marginparsep=.25in]{geometry} % Sets page geometry
\usepackage{fancyhdr} % Adds control of headers and footers
%%%%  Testing
\usepackage{pifont}
\usepackage{anyfontsize}
\usepackage{multicol}
%%%%%%%%%%%%%


%\newcommand{\localtextbulletone}{\textcolor{HeaderOrangeC}{\raisebox{.45ex}{\rule{.6ex}{.6ex}}}}

% ding characters specified in symbols-a4 pg12
\newcommand{\localtextbulletone}{\textcolor{HeaderOrangeC}{\ding{227}}}
\renewcommand{\labelitemi}{\localtextbulletone}

\newcommand{\HRule}{\rule{\linewidth}{0.5mm}} % Command to make horizontal graphic lines
\newcommand{\marginText}{\fontfamily{cmr}\fontseries{b}\color{red}\fontsize{9}{12}\selectfont}
\newcommand{\helv}{\fontfamily{phv}\fontseries{b}\fontsize{9}{11}\selectfont}
% font has encoding, family,  series{b}(bold), shape, size{(size)}{(baselineskip)}
%       % https://www.latex-project.org/help/documentation/fntguide.pdf
%       % https://www.overleaf.com/learn/latex/Font_typefaces
%
\graphicspath{{img/}{GIS_ChampionSection/img/}{awardsChapter/GIS_ChampionSection/img/}{brandPart/awardsChapter/GIS_ChampionSection/img/}{img/}{pairedProgSection/img/}{methodChapter/pairedProgSection/img/}{methodPart/methodChapter/pairedProgSection/img/}{documentationSection/img/}{methodChapter/documentationSection/img/}{methodPart/methodChapter/documentationSection/img/}{docStorageOrgSection/img/}{methodChapter/docStorageOrgSection/img/}{methodPart/methodChapter/docStorageOrgSection/img/}{QGisSection/img/}{toolsChapter/QGisSection/img/}{servicePart/toolsChapter/QGisSection/img/}{ESRISection/img/}{toolChapter/ESRISection/img/}{servicePart/toolChapter/ESRISection/img/}{../../../../source/}{../../source/}{servicePart/applicationsChapter/treasurerSection/img/}{servicePart/toolsChapter/gisAdminSection/img/}{gisAdminSection/img/}{servicePart/toolsChapter/bsaSupportSection/img/}{servicePart/toolsChapter/coreDataSection/img/}{../img}}
%
%%%%%%%%%%%%%%%%%%%%%%%%%%%%%%%%%%%%%%%%%%%%%%%%%%%%%%%%%%%%%
%\usepackage{fancyhdr} % Adds control of headers and footers
%
%     Code Key: E : Even page, O : Odd page, L Left field,  C : Center field,
%               R : Right field, H : Header, F : Footer
%    \leftmark : Contains the Left argument of the last \markboth on the pages
%    \rightmark : Contains the Right Argument of the first \markboth of the page OR:
%                 the first /markright of the page
%
\renewcommand{\chaptermark}[1] % Redefinition of how chapter mark is displayed
{\markboth{\MakeUppercase{\thechapter.\ #1}}{}}
%
\renewcommand{\subsectionmark}[1] % Redefinition of how subsection mark is displayed
{\markright{\MakeUppercase{\thesubsection.\ #1}}}
%
\renewcommand{\headrulewidth}{0.5pt} % Line after header
\renewcommand{\footrulewidth}{0.5pt} % line before footer
%
\fancyhf{} % clear all header and footer fields
%
\fancyhead[LE,RO]{\helv \thepage} % Left Even and Right Odd headers
%                                 % \thepage macro displays the current page number
\fancyhead[LO]{\helv \leftmark} % Left Odd headers are set to /markboth
%                               % (chapters from \renewcommand{\chaptermark}[1]... )
\fancyhead[RE]{\helv \rightmark} % Right Even Headers
%                                % (subsection from \renewcommand{\subsectionmark}[1]...)




%\fancyheadoffset[RE,LO,RO,LE]{.5\textwidth}
%

% Alternate fancy header options
%\renewcommand{\sectionmark}[1] % Redefinition of how section mark is displayed
%{\markright{\MakeUppercase{\thesection.\ #1}}}
%
% Alternate fancy footer options
%\fancyfoot[C]{\textbf{\thepage}} % Page number on bottom center
%\fancyfoot[EC,OC]{\helv \thesection} % Section number bottom center

%%%%%%%%%%%%%%%%%%%%%%%%%%%%%%%%%%%%%%%%%%%%%%%%%%%%%%%%%%
% Test area

%\setlength\parindent{0pt} % eliminates indents
%\tolerance=1500 % Number affects overfull and Underful box warnings
%\usepackage{caption}

%\setlist[description]{leftmargin=\parindent,labelindent=\parindent}
\newcommand{\titleGeometry}{\geometry{top=1in,bottom=1in,inner=1.5in,outer=1.5in}} % Sets geometry for first part}

\usepackage[pdftex,breaklinks,colorlinks=true,linkcolor=black,citecolor=blue,urlcolor=red,linktocpage=false,pagebackref=true,filecolor=magenta]{hyperref}% http://www.tug.org/applications/hyperref/manual.html#x1-100003.6
\hypersetup{pdfauthor={Allegan County GIS Services},% Page 248 in Latex Beinners Guide
pdftitle={What We Do},
pdfsubject={Documentation of everything},
pdfkeywords={documentation,gis}}

\end{verbatim}
  %
Add the line:
  %
\begin{verbatim}
\newglossaryentry{ex}{name={sample},description={an example}}


\newglossaryentry{projection}{name={map projection},description={Representing a sphere on a flat surface}}
\end{verbatim}
  %
\subparagraph[To use a glossary term]{To use a glossary term in the subdocument:\texorpdfstring{\\}{}}
In place of the term, use code referencing the key (in the glossaryEntries file):
  %
\begin{itemize}
\item \textbackslash gls\{key\}
\end{itemize}
  %
\subparagraph{To add the glossary to the subdocument:}
  %
\begin{itemize}
\item Add the line \textbackslash makeglossaries to the preamble of the subdocument.
\item Add the line \textbackslash printglossaries to the subdocument.
\item Run makeglossaries in command line on the subdocument similar to how is described above.
\end{itemize}
  %
\clearpage

\subsubsection[Using The Bibliography(References)]{{Using The Bibliography(References)}}
  %
\paragraph{Bibliography requirements}
Compiling the bibliography requires running bibtex either in a \LaTeX{} IDE or in command line as described here.  PDFLatex must be run first to create a .aux file that is used by bibtex to create a .bbl file.  After the .bbl file is created, PDFLatex must be run again to insert the bibliography at the \textbackslash bibliography location.
  %
\paragraph{Inserting the bibliography}
In the \LaTeX{} code:
\begin{verbatim} \bibliography\{referenceEntries}\end{verbatim}
{\smallbtn Inserts a bibliography called referenceEntries.bib from the same folder as the project .aux file}
  %
\paragraph{Creating a new bibliography entry}
To \textbf{create a new bibliography entry:} Add an entry to referenceEntries.bib.  Save it there and then use bibtex to recompile the .bbl file.
  %
\paragraph{Rebuilding the bibliography}
\textbf{To Recompile the .bbl}.  In the (main document)build folder:
  %
\begin{itemize}
\item Launch command prompt
\item Enter command: \textbf{{\large bibtex GISDocumentation}}
\end{itemize}
  %
\noindent{\smallbtn Note that this command reads the .aux file and creates the .bbl file.  The .aux file is created by compiling with PDFLatex.  If there is no .aux file the command will fail}
  %
\paragraph{To cite a bibliography source in a subdocument}
In the place that you want the citation In the \LaTeX{} code:
  %
\begin{verbatim} ~\cite[pg.#]{key} \end{verbatim}
  %
\clearpage

\subsubsection[Using The Index]{{Using The Index}}
  %
\paragraph{Index requirements:}
Compiling the index requires running the makeindex command either in a \LaTeX{} IDE or in command line as described here.  PDFLatex must be run first to create a .aux file that is used by makeindex to create an .idx file.  After the .idx file is created, PDFLatex must be run again to insert the index at the \textbackslash printindex location.
  %
\paragraph{Creating a new index entry}
To \textbf{create a new index entry:} Add an entry to indexEntries.tex.  Save it there and then use the makeindex command to recompile the .idx file.

\paragraph{Rebuilding the index}

\textbf{To Recompile the .idx} In the (main document)build folder:
  %
\begin{itemize}
\item Launch command prompt
\item enter command: \textbf{{\large makeindex GISDocumentation*}}
\end{itemize}
  %
\noindent{\smallbtn Note that this command reads the .aux file and creates the .idx file.  The .aux file is created by compiling with PDFLatex.  If there is no .aux file the command will fail. Run PDFLatex first}
  %
\paragraph{Access the index from a subdocument}
In the subdocument you must add code to input the indexEntries file.  For example:\\
After the line:
  %
\begin{verbatim}


%%%%%%%%%%%%%%%%%%%%%%%%%%%%%%%%%%%%%%%%%%%%%%%%%%%%%%%%%%%%%
% Tried and True Preamble Section
\usepackage{import} % Required for importing other .tex docs.  (import uses everything bw Begin and End Doc)
\usepackage{float} % Required for specifying the exact location of a figure or table
\usepackage{wrapfig} % Provides environment to wrap figures with text
\usepackage{graphicx} % Required for including images
\usepackage{cite} % improves citation handling
%\usepackage[toc,title,page]{appendix}
\usepackage{pdfpages} % enables loading a pdf into the doc
\usepackage{makeidx} % Enables index features
\usepackage{glossaries} % must be after hyperref
\usepackage{blindtext} % enables ipsum loerem "dummy" text blocks
\usepackage{enumitem} % enables control over enumerate, itemize, and description
\usepackage[dvipsnames]{xcolor}%black, blue, brown, cyan, darkgray, gray, green, lightgray, lime, magenta, olive, orange, pink, purple, red, teal, violet, white, yellow.
%
\definecolor{HeaderBlueA1}{RGB}{10, 91, 163}
\definecolor{HeaderBlueA2}{RGB}{101, 98, 174}
\definecolor{HeaderBlueA3}{RGB}{84, 125, 161}
\definecolor{HeaderBlueB}{RGB}{53, 98, 138}
\definecolor{HeaderBlueC}{RGB}{31,78,121}  % HeaderBlue1
\definecolor{HeaderBlueD}{RGB}{16, 60, 99}
\definecolor{HeaderBlueE}{RGB}{4, 40, 72 }

\definecolor{HeaderOrangeA}{RGB}{249, 161, 121}
\definecolor{HeaderOrangeB}{RGB}{214, 116, 72}
\definecolor{HeaderOrangeC}{RGB}{187, 84, 37}
\definecolor{HeaderOrangeD}{RGB}{152, 57, 14}
\definecolor{HeaderOrangeE}{RGB}{111, 35, 0}

\definecolor{HeaderGoldA}{RGB}{249, 218, 121}
\definecolor{HeaderGoldB}{RGB}{214, 180, 72}
\definecolor{HeaderGoldC}{RGB}{187, 151, 37}
\definecolor{HeaderGoldD}{RGB}{152, 119, 14}
\definecolor{HeaderGoldE}{RGB}{111, 84, 0}

% Color triad based on HeaderBlue1
% http://paletton.com/
% Blues:
% 84, 125, 161 lightest blue
% 53, 98, 138 light blue
% 31,78,121 HeaderBlue1
% 16, 60, 99 darker blue
% 4, 40, 72 darkest blue
% Oranges:
% 249, 161, 121 lightest orange
% 214, 116, 72 light orange
% 187, 84, 37 main orange
% 152, 57, 14 darker orange
% 111, 35, 0 dark orange
% Golds:
% 249, 218, 121 lightest goldenrod
% 214, 180, 72 light goldenrod
% 187, 151, 37 goldenrod
% 152, 119, 14 dark goldenrod
% 111, 84, 0 darkest goldenrod

\definecolor{HyperlinkBlue1}{RGB}{64,172,209}
\definecolor{graphicOrange}{RGB}{255,153,51}
% Header1 is Arial(body)(Caps) 14pt
% Header2 is Arial(body) 12pt
% Body is arial 9pt
\usepackage{marginnote}
%\usepackage{geometry}
\usepackage[top=1.5in, bottom=1.5in, outer=2.5in, inner=1.25in, heightrounded, marginparwidth=1.5in, marginparsep=.25in]{geometry} % Sets page geometry
\usepackage{fancyhdr} % Adds control of headers and footers
%%%%  Testing
\usepackage{pifont}
\usepackage{anyfontsize}
\usepackage{multicol}
%%%%%%%%%%%%%


%\newcommand{\localtextbulletone}{\textcolor{HeaderOrangeC}{\raisebox{.45ex}{\rule{.6ex}{.6ex}}}}

% ding characters specified in symbols-a4 pg12
\newcommand{\localtextbulletone}{\textcolor{HeaderOrangeC}{\ding{227}}}
\renewcommand{\labelitemi}{\localtextbulletone}

\newcommand{\HRule}{\rule{\linewidth}{0.5mm}} % Command to make horizontal graphic lines
\newcommand{\marginText}{\fontfamily{cmr}\fontseries{b}\color{red}\fontsize{9}{12}\selectfont}
\newcommand{\helv}{\fontfamily{phv}\fontseries{b}\fontsize{9}{11}\selectfont}
% font has encoding, family,  series{b}(bold), shape, size{(size)}{(baselineskip)}
%       % https://www.latex-project.org/help/documentation/fntguide.pdf
%       % https://www.overleaf.com/learn/latex/Font_typefaces
%
\graphicspath{{img/}{GIS_ChampionSection/img/}{awardsChapter/GIS_ChampionSection/img/}{brandPart/awardsChapter/GIS_ChampionSection/img/}{img/}{pairedProgSection/img/}{methodChapter/pairedProgSection/img/}{methodPart/methodChapter/pairedProgSection/img/}{documentationSection/img/}{methodChapter/documentationSection/img/}{methodPart/methodChapter/documentationSection/img/}{docStorageOrgSection/img/}{methodChapter/docStorageOrgSection/img/}{methodPart/methodChapter/docStorageOrgSection/img/}{QGisSection/img/}{toolsChapter/QGisSection/img/}{servicePart/toolsChapter/QGisSection/img/}{ESRISection/img/}{toolChapter/ESRISection/img/}{servicePart/toolChapter/ESRISection/img/}{../../../../source/}{../../source/}{servicePart/applicationsChapter/treasurerSection/img/}{servicePart/toolsChapter/gisAdminSection/img/}{gisAdminSection/img/}{servicePart/toolsChapter/bsaSupportSection/img/}{servicePart/toolsChapter/coreDataSection/img/}{../img}}
%
%%%%%%%%%%%%%%%%%%%%%%%%%%%%%%%%%%%%%%%%%%%%%%%%%%%%%%%%%%%%%
%\usepackage{fancyhdr} % Adds control of headers and footers
%
%     Code Key: E : Even page, O : Odd page, L Left field,  C : Center field,
%               R : Right field, H : Header, F : Footer
%    \leftmark : Contains the Left argument of the last \markboth on the pages
%    \rightmark : Contains the Right Argument of the first \markboth of the page OR:
%                 the first /markright of the page
%
\renewcommand{\chaptermark}[1] % Redefinition of how chapter mark is displayed
{\markboth{\MakeUppercase{\thechapter.\ #1}}{}}
%
\renewcommand{\subsectionmark}[1] % Redefinition of how subsection mark is displayed
{\markright{\MakeUppercase{\thesubsection.\ #1}}}
%
\renewcommand{\headrulewidth}{0.5pt} % Line after header
\renewcommand{\footrulewidth}{0.5pt} % line before footer
%
\fancyhf{} % clear all header and footer fields
%
\fancyhead[LE,RO]{\helv \thepage} % Left Even and Right Odd headers
%                                 % \thepage macro displays the current page number
\fancyhead[LO]{\helv \leftmark} % Left Odd headers are set to /markboth
%                               % (chapters from \renewcommand{\chaptermark}[1]... )
\fancyhead[RE]{\helv \rightmark} % Right Even Headers
%                                % (subsection from \renewcommand{\subsectionmark}[1]...)




%\fancyheadoffset[RE,LO,RO,LE]{.5\textwidth}
%

% Alternate fancy header options
%\renewcommand{\sectionmark}[1] % Redefinition of how section mark is displayed
%{\markright{\MakeUppercase{\thesection.\ #1}}}
%
% Alternate fancy footer options
%\fancyfoot[C]{\textbf{\thepage}} % Page number on bottom center
%\fancyfoot[EC,OC]{\helv \thesection} % Section number bottom center

%%%%%%%%%%%%%%%%%%%%%%%%%%%%%%%%%%%%%%%%%%%%%%%%%%%%%%%%%%
% Test area

%\setlength\parindent{0pt} % eliminates indents
%\tolerance=1500 % Number affects overfull and Underful box warnings
%\usepackage{caption}

%\setlist[description]{leftmargin=\parindent,labelindent=\parindent}
\newcommand{\titleGeometry}{\geometry{top=1in,bottom=1in,inner=1.5in,outer=1.5in}} % Sets geometry for first part}

\usepackage[pdftex,breaklinks,colorlinks=true,linkcolor=black,citecolor=blue,urlcolor=red,linktocpage=false,pagebackref=true,filecolor=magenta]{hyperref}% http://www.tug.org/applications/hyperref/manual.html#x1-100003.6
\hypersetup{pdfauthor={Allegan County GIS Services},% Page 248 in Latex Beinners Guide
pdftitle={What We Do},
pdfsubject={Documentation of everything},
pdfkeywords={documentation,gis}}

\end{verbatim}
  %
Add the line:
  %
\begin{verbatim}
% add index entries here and in the text for the index to track them
\index{map projections}
\index{ESRI Product documentation}
\index{functionality matrix}
\index{ArcMap 10.5 functionality matrix}
\index{ArcGIS Enterprise 10.5 functionality matrix}
\end{verbatim}
  %
\subparagraph[Using an index term]{To use a index term in the subdocument:}
In place of the term, use code referencing the key (in the indexEntries file):
  %
\begin{itemize}
\item \textbackslash index \{key\}
\end{itemize}
  %
\subparagraph{To add the index to the subdocument:}
  %
\begin{itemize}
\item Add the line \textbackslash makeindex to the preamble of the subdocument.
\item Add the line \textbackslash printindex to the subdocument.
\item Run makeindex in command line on the subdocument similar to how is described above.
\end{itemize}
  %
\subsubsection[Using the Appendices]{{Using the Appendices}}

\end{document}

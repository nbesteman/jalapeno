% Requires Memoir Class
%%%%%%%%%%%%%%%%%%%%%%%%%%%%%%%%%%%%%%%%%%%%%%%%%%%%%%%%%%%%
% HEADERS AND FOOTERS
    %  Page Styles
    %  In LATEX the page style refers to the part of the page building mechanism that
    %   attaches the headers and footers to the actual page.
%+++++++++++++++++++++++++++++++++++++++++++++++++++++++++++++++++++
%  DEFINE JALAPENO PAGE STYLE (JalapenopageStyleA page style)
\makepagestyle{jalapenoPageStyleA}
    % HEADERS
\makeheadrule {jalapenoPageStyleA}{\textwidth}{\normalrulethickness}
\makeheadfootruleprefix{jalapenoPageStyleA}{\color{HeaderOrangeE}}{\color{HeaderOrangeE}}
\makeevenhead {jalapenoPageStyleA}{\helv \thepage}{} {\helv \rightmark}
\makeoddhead {jalapenoPageStyleA}{\helv \leftmark}{}{\helv \thepage}
%\makeevenhead {jalapenoPageStyleA}{\helv \thepage}{}
%\makeoddhead {jalapenoPageStyleA}{}{\helv \thepage}
    % FOOTERS
\makefootrule {jalapenoPageStyleA}{\textwidth}{\normalrulethickness}{\footruleskip}
\makeevenfoot {jalapenoPageStyleA}{}{} {}
\makeoddfoot {jalapenoPageStyleA}{}{} {}
%\makeevenfoot {jalapenoPageStyleA}{}{\thepage} {}
%\makeoddfoot {jalapenoPageStyleA}{}{\thepage} {}
    % MARKS
      % NOTE: See page 11 of Page Styles in Memoir
\makeatletter % because of \@chapapp
  %
\makepsmarks {jalapenoPageStyleA}{
  %\nouppercaseheads
\createmark {chapter} {both} {shownumber}{\@chapapp\ }{. \ } % set leftmark to chapter
%\createmark {section} {right}{shownumber}{} {. \ }
\createmark {subsection} {right}{shownumber}{} {. \ } % set rightmark to subsection
%\createmark {subsubsection}{right}{shownumber}{} {. \ } % set rightmark to subsubsection
\createplainmark {toc} {both} {\contentsname}
\createplainmark {lof} {both} {\listfigurename}
\createplainmark {lot} {both} {\listtablename}
\createplainmark {bib} {both} {\bibname}
\createplainmark {index} {both} {\indexname}
\createplainmark {glossary} {both} {\glossaryname}
}
\makeatother
%-------------------------------------------------------------------
  %
%+++++++++++++++++++++++++++++++++++++++++++++++++++++++++++++++++++
% set page number color for plain page
\makeevenfoot {plain}{}{\helv \thepage} {}
\makeoddfoot {plain}{}{\helv \thepage} {}

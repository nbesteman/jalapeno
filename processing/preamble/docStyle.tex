\makeatletter % handler for @
    %
\makechapterstyle{customchpstyle}{%
    %
    \newlength{\chapColorbarheight}
    \setlength{\chapColorbarheight}{.06in}		% Setting the height of the bar
    %
    \newlength{\chapnumboxlength}
    \setlength{\chapnumboxlength}{.75in}	% Setting the length of the box containing chapter number
    %
    \newlength{\leftbarlength}
    \setlength{\leftbarlength}{.5in}  		% Setting length of the bar left to the chapter number
    \setlength{\afterchapskip}{.5in}
    \renewcommand*{\chapterheadstart}{\vspace*{.25in}} % vspace from top of page
    \renewcommand*{\afterchapternum}{\vspace*{-.9in}} % neg moves chapter title above chapter number
    \renewcommand*{\chapnumfont}{\slshape} % Number Font
    \renewcommand*{\chaptitlefont}{\flushright\color{HeaderOrangeE}\slshape\huge}
    \renewcommand*{\printchaptername}{} % sets the text in brackets before the number
    %\makeatletter
    \newcommand*{\thickrulefill}{\leavevmode \leaders \hrule height \chapColorbarheight \hfill \kern \z@} % rightbar length setter
    %\makeatother
    \renewcommand*{\printchapternum}{%
        %\makebox[0pt][l]{%
        \color{HeaderOrangeE}
        \rule{\leftbarlength}{\chapColorbarheight} % Left bar
        \quad % hspace
        \resizebox{!}{\chapnumboxlength}
        {\chapnumfont \thechapter} % Chapter name and number
        \quad % hspace
        \color{HeaderOrangeE}\thickrulefill % implements thickrulefill
     } %
     \makeatother % closes handler for @
} % End of customchpstyle
    %
    %\setsecnumdepth{subsection} % turns on subsec numbering
    %
\setsecheadstyle{\scshape\color{HeaderOrangeA}} % default is \Large\bfseries
    %\setsecheadstyle{\scshape\color{HeaderOrangeB}}
    %
    % Subsection
\setsubsecheadstyle{\large\scshape\color{HeaderOrangeB}} % default is \large\bfseries
    %\setsubsecheadstyle{\huge\centering\color{HeaderOrangeE}}%
    % Subsubsection
\setsubsubsecheadstyle{\Huge\scshape\centering\color{HeaderBlueE}} % default is \bfseries
    %\setsubsubsecheadstyle{\LARGE\scshape\centering\color{HeaderBlueC}}%   1
    % Paragraph
\setparaheadstyle{\bfseries\Large\color{HeaderBlueC}}%
     %\setparaheadstyle{\Large\color{HeaderBlueA}}%2
     % Subparagraph
     %\setsubparaheadstyle{\large\color{HeaderBlueB}}%
\setsubparaheadstyle{\bfseries\large\color{HeaderOrangeB}}%
     %
     % \setbeforeSskip{<skip>} pg .95
     % The absolute value of the <skip> length argument is the space to leave above the heading.
     % If the actual value is negative then the first line after the heading will not be indented.
\setbeforesecskip{-35pt plus -10pt minus -2pt}
\setbeforesubsecskip{-32.5pt plus -10pt minus -2pt}
\setbeforesubsubsecskip{-32.5pt plus -10pt minus -2pt}
\setbeforeparaskip{-32.5pt plus -10pt minus -2pt}
\setbeforesubparaskip{-5pt plus -3pt minus -1pt}
    %
    % \setSindent{<length>}
    % The value of the length argument is the indentation of the heading (number and
    % title) from the lefthand margin. This is normally 0pt.
\setsubparaindent{0pt}
    % \setafterSskip{<skip>} pg.96
    % If the value of the <skip> length argument is positive it is the space to leave between the
    % display heading and the following text. If it is negative, then the heading will be runin
    % and the value is the horizontal space between the end of the heading and the following text.
    %
\setaftersecskip{.05in}
\setaftersubsecskip{.15in}
\setaftersubsubsecskip{.1in}
\setafterparaskip{.05in}
\setaftersubparaskip{.05in}
    %
\renewcommand{\partnamefont}{\color{HeaderOrangeE}\Huge\normalfont}
\renewcommand{\partnumfont}{\color{HeaderOrangeE}\Huge\normalfont}

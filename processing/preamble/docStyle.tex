\makeatletter

\setsecheadstyle{\scshape\color{HeaderOrangeB}}
    % Subsection
%\setsubsecheadstyle{\begin{mdframed}[backgroundcolor=HeaderGoldD]\huge\centering\color{HeaderOrangeE}}%
\setsubsecheadstyle{\huge\centering\color{HeaderOrangeE}}%
    % Subsubsection
\setsubsubsecheadstyle{\LARGE\scshape\centering\color{HeaderBlueC}}%
    % Paragraph
\setparaheadstyle{\Large\color{HeaderBlueA1}}%
    % Subparagraph
\setsubparaheadstyle{\large\color{HeaderOrangeB}}%
    %
    % \setafterSskip{<skip>} pg.96
    % If the value of the <skip> length argument is positive it is the space to leave between the
    % display heading and the following text. If it is negative, then the heading will be runin
    % and the value is the horizontal space between the end of the heading and the following text.
\setaftersecskip{5pt}
\setaftersubsecskip{5pt}
\setaftersubsubsecskip{5pt}
\setafterparaskip{5pt}
\setaftersubparaskip{3pt}
    % \setbeforeSskip{<skip>} pg .95
    % The absolute value of the <skip> length argument is the space to leave above the heading.
    % If the actual value is negative then the first line after the heading will not be indented.
\setbeforesecskip{-35pt plus -10pt minus -2pt}
\setbeforesubsecskip{-32.5pt plus -10pt minus -2pt}
\setbeforesubsubsecskip{-32.5pt plus -10pt minus -2pt}
\setbeforeparaskip{-32.5pt plus -10pt minus -2pt}
\setbeforesubparaskip{-15pt plus -3pt minus -1pt}

\newlength{\chapColorbarheight}
\setlength{\chapColorbarheight}{.08in}		% Setting the height of the bar
%
\newlength{\chapnumboxlength}
\setlength{\chapnumboxlength}{.75in}	% Setting the length of the box containing chapter number
%
\newlength{\leftbarlength}
\setlength{\leftbarlength}{.5in}  		% Setting length of the bar left to the chapter number
%
\newlength{\rightbarlength}          % Setting length of the bar right of the chapter number
%\setlength{\rightbarlength}{\textwidth}
\setlength{\rightbarlength}{3in}
%\addtolength{\rightbarlength}{-\leftbarlength} % trimming it back to the margin
%\addtolength{\rightbarlength}{-20pt}
%
\makechapterstyle{customchpstyle}{%
   \setlength{\afterchapskip}{.5in}
   \renewcommand*{\chapterheadstart}{\vspace*{.25in}} % vspace from top of page
   \renewcommand*{\afterchapternum}{\vspace*{-1in}} % neg moves chapter title above chapter number
   \renewcommand*{\chapnamefont}{\normalfont\Large\flushright} % Name Font
   \renewcommand*{\chaptitlefont}{\scshape\huge\bfseries\flushright\color{chapColor}}
   \renewcommand*{\printchaptername}{} % sets the text in brackets before the number
   %
   \newcommand{\thickrulefill}{\leavevmode \leaders \hrule height \chapColorbarheight \hfill \kern \z@}
   \renewcommand*{\chapnumfont}{\slshape\huge} % Number Font
    %
   \renewcommand*{\printchapternum}{%
       %\makebox[0pt][l]{%
       \color{chapColor}
       \rule{\leftbarlength}{\chapColorbarheight} % Left bar
       %\quad % vskip
       \resizebox{!}{\chapnumboxlength}
       {\chapnumfont \thechapter} % Chapter name and number
       \quad % vskip
       %\color{chapColor}
       %\rule{\rightbarlength}{\chapColorbarheight}
       \color{chapColor}\thickrulefill % implements thickrulefill
     }%
   }%
}% End of customchpstyle

\chapterstyle{customchpstyle}

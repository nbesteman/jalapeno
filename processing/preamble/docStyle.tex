\makeatletter
\makechapterstyle{customchpstyle}{%
   \setlength{\afterchapskip}{20pt}
   \renewcommand*{\chapterheadstart}{\vspace*{10pt}}
   \renewcommand*{\afterchapternum}{\par\nobreak\vskip 25pt}
   \renewcommand*{\chapnamefont}{\normalfont\Large\flushleft}
   \renewcommand*{\chapnumfont}{\slshape\huge}
   \renewcommand*{\chaptitlefont}{\scshape\huge\bfseries\flushleft\color{chapColor}}
   \renewcommand*{\printchaptername}{} % sets the text in brackets before the number
   %\renewcommand*{\chapternamenum}{}
   \newcommand{\thickrulefill}{\leavevmode \leaders \hrule height \chapColorbarheight \hfill \kern \z@}

   \renewcommand*{\printchapternum}{%
     \makebox[0pt][l]{%
       \color{chapColor}\rule{\leftbarlength}{\chapColorbarheight}											% Left bar
       \resizebox{\chapnumboxlength}{!}{\color{chapColor}\chapnumfont\thechapter}	% Chapter number
       \color{chapColor}\rule{\rightbarlength}{\chapColorbarheight}											% Right bar
     }%
   }%
   \newlength{\chapColorbarheight}
   \setlength{\chapColorbarheight}{2.5pt}		% Setting the height of the bar

   \newlength{\chapnumboxlength}
   \setlength{\chapnumboxlength}{48pt}	% Setting the length of the box containing chapter number

   \newlength{\leftbarlength}
   \setlength{\leftbarlength}{36pt}  		% Setting length of the bar left to the chapter number

   \newlength{\rightbarlength}
   \setlength{\rightbarlength}{\textwidth}
   %
   %\addtolength{\rightbarlength}{-\leftbarlength}
   %\addtolength{\rightbarlength}{-\chapnumboxlength}	% Adjusting the right bar to fit the text margin
   %\addtolength{\rightbarlength}{-0.5em} % "By eye" adjustment.. - shouldn't really be there :(

    %\makeoddfoot{plain}{}{}{\thepage}
    }%


    %  \newcommand{command}[arguments][optional]{definition}
    %  page 44 of Latex Beginners Guide
    %
    % heading presets
    %\headstyles{dowding}

\setsecheadstyle{\scshape\color{HeaderOrangeB}}
    % Subsection
\setsubsecheadstyle{\huge\centering\color{HeaderOrangeE}}%
    % Subsubsection
\setsubsubsecheadstyle{\LARGE\scshape\centering\color{HeaderBlueC}}%
    % Paragraph
\setparaheadstyle{\Large\color{HeaderBlueA1}}%
    % Subparagraph
\setsubparaheadstyle{\large\color{HeaderOrangeB}}%
    %
    % \setafterSskip{<skip>} pg.96
    % If the value of the <skip> length argument is positive it is the space to leave between the
    % display heading and the following text. If it is negative, then the heading will be runin
    % and the value is the horizontal space between the end of the heading and the following text.
\setaftersecskip{5pt}
\setaftersubsecskip{5pt}
\setaftersubsubsecskip{5pt}
\setafterparaskip{5pt}
\setaftersubparaskip{3pt}
    % \setbeforeSskip{<skip>} pg .95
    % The absolute value of the <skip> length argument is the space to leave above the heading.
    % If the actual value is negative then the first line after the heading will not be indented.
\setbeforesecskip{-35pt plus -10pt minus -2pt}
\setbeforesubsecskip{-32.5pt plus -10pt minus -2pt}
\setbeforesubsubsecskip{-32.5pt plus -10pt minus -2pt}
\setbeforeparaskip{-32.5pt plus -10pt minus -2pt}
\setbeforesubparaskip{-15pt plus -3pt minus -1pt}
    %
%\renewcommand{\midchapskip}{10pt} % defines vspace bw <Chapter#> and <Chapter Title> default is 20pt
    %
%\renewcommand{\afterchapskip}{20pt} % defines vspace bw <Chapter Title> and <Section Heading> defauklt uis 40pt
    %

\chapterstyle{customchpstyle}
